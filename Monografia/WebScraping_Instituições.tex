
    




    
\documentclass[11pt]{article}

    
    \usepackage[breakable]{tcolorbox}
    \tcbset{nobeforeafter} % prevents tcolorboxes being placing in paragraphs
    \usepackage{float}
    \floatplacement{figure}{H} % forces figures to be placed at the correct location
    
    \usepackage[T1]{fontenc}
    % Nicer default font (+ math font) than Computer Modern for most use cases
    \usepackage{mathpazo}

    % Basic figure setup, for now with no caption control since it's done
    % automatically by Pandoc (which extracts ![](path) syntax from Markdown).
    \usepackage{graphicx}
    % We will generate all images so they have a width \maxwidth. This means
    % that they will get their normal width if they fit onto the page, but
    % are scaled down if they would overflow the margins.
    \makeatletter
    \def\maxwidth{\ifdim\Gin@nat@width>\linewidth\linewidth
    \else\Gin@nat@width\fi}
    \makeatother
    \let\Oldincludegraphics\includegraphics
    % Set max figure width to be 80% of text width, for now hardcoded.
    \renewcommand{\includegraphics}[1]{\Oldincludegraphics[width=.8\maxwidth]{#1}}
    % Ensure that by default, figures have no caption (until we provide a
    % proper Figure object with a Caption API and a way to capture that
    % in the conversion process - todo).
    \usepackage{caption}
    \DeclareCaptionLabelFormat{nolabel}{}
    \captionsetup{labelformat=nolabel}

    \usepackage{adjustbox} % Used to constrain images to a maximum size 
    \usepackage{xcolor} % Allow colors to be defined
    \usepackage{enumerate} % Needed for markdown enumerations to work
    \usepackage{geometry} % Used to adjust the document margins
    \usepackage{amsmath} % Equations
    \usepackage{amssymb} % Equations
    \usepackage{textcomp} % defines textquotesingle
    % Hack from http://tex.stackexchange.com/a/47451/13684:
    \AtBeginDocument{%
        \def\PYZsq{\textquotesingle}% Upright quotes in Pygmentized code
    }
    \usepackage{upquote} % Upright quotes for verbatim code
    \usepackage{eurosym} % defines \euro
    \usepackage[mathletters]{ucs} % Extended unicode (utf-8) support
    \usepackage[utf8x]{inputenc} % Allow utf-8 characters in the tex document
    \usepackage{fancyvrb} % verbatim replacement that allows latex
    \usepackage{grffile} % extends the file name processing of package graphics 
                         % to support a larger range 
    % The hyperref package gives us a pdf with properly built
    % internal navigation ('pdf bookmarks' for the table of contents,
    % internal cross-reference links, web links for URLs, etc.)
    \usepackage{hyperref}
    \usepackage{longtable} % longtable support required by pandoc >1.10
    \usepackage{booktabs}  % table support for pandoc > 1.12.2
    \usepackage[inline]{enumitem} % IRkernel/repr support (it uses the enumerate* environment)
    \usepackage[normalem]{ulem} % ulem is needed to support strikethroughs (\sout)
                                % normalem makes italics be italics, not underlines
    \usepackage{mathrsfs}
    

    
    % Colors for the hyperref package
    \definecolor{urlcolor}{rgb}{0,.145,.698}
    \definecolor{linkcolor}{rgb}{.71,0.21,0.01}
    \definecolor{citecolor}{rgb}{.12,.54,.11}

    % ANSI colors
    \definecolor{ansi-black}{HTML}{3E424D}
    \definecolor{ansi-black-intense}{HTML}{282C36}
    \definecolor{ansi-red}{HTML}{E75C58}
    \definecolor{ansi-red-intense}{HTML}{B22B31}
    \definecolor{ansi-green}{HTML}{00A250}
    \definecolor{ansi-green-intense}{HTML}{007427}
    \definecolor{ansi-yellow}{HTML}{DDB62B}
    \definecolor{ansi-yellow-intense}{HTML}{B27D12}
    \definecolor{ansi-blue}{HTML}{208FFB}
    \definecolor{ansi-blue-intense}{HTML}{0065CA}
    \definecolor{ansi-magenta}{HTML}{D160C4}
    \definecolor{ansi-magenta-intense}{HTML}{A03196}
    \definecolor{ansi-cyan}{HTML}{60C6C8}
    \definecolor{ansi-cyan-intense}{HTML}{258F8F}
    \definecolor{ansi-white}{HTML}{C5C1B4}
    \definecolor{ansi-white-intense}{HTML}{A1A6B2}
    \definecolor{ansi-default-inverse-fg}{HTML}{FFFFFF}
    \definecolor{ansi-default-inverse-bg}{HTML}{000000}

    % commands and environments needed by pandoc snippets
    % extracted from the output of `pandoc -s`
    \providecommand{\tightlist}{%
      \setlength{\itemsep}{0pt}\setlength{\parskip}{0pt}}
    \DefineVerbatimEnvironment{Highlighting}{Verbatim}{commandchars=\\\{\}}
    % Add ',fontsize=\small' for more characters per line
    \newenvironment{Shaded}{}{}
    \newcommand{\KeywordTok}[1]{\textcolor[rgb]{0.00,0.44,0.13}{\textbf{{#1}}}}
    \newcommand{\DataTypeTok}[1]{\textcolor[rgb]{0.56,0.13,0.00}{{#1}}}
    \newcommand{\DecValTok}[1]{\textcolor[rgb]{0.25,0.63,0.44}{{#1}}}
    \newcommand{\BaseNTok}[1]{\textcolor[rgb]{0.25,0.63,0.44}{{#1}}}
    \newcommand{\FloatTok}[1]{\textcolor[rgb]{0.25,0.63,0.44}{{#1}}}
    \newcommand{\CharTok}[1]{\textcolor[rgb]{0.25,0.44,0.63}{{#1}}}
    \newcommand{\StringTok}[1]{\textcolor[rgb]{0.25,0.44,0.63}{{#1}}}
    \newcommand{\CommentTok}[1]{\textcolor[rgb]{0.38,0.63,0.69}{\textit{{#1}}}}
    \newcommand{\OtherTok}[1]{\textcolor[rgb]{0.00,0.44,0.13}{{#1}}}
    \newcommand{\AlertTok}[1]{\textcolor[rgb]{1.00,0.00,0.00}{\textbf{{#1}}}}
    \newcommand{\FunctionTok}[1]{\textcolor[rgb]{0.02,0.16,0.49}{{#1}}}
    \newcommand{\RegionMarkerTok}[1]{{#1}}
    \newcommand{\ErrorTok}[1]{\textcolor[rgb]{1.00,0.00,0.00}{\textbf{{#1}}}}
    \newcommand{\NormalTok}[1]{{#1}}
    
    % Additional commands for more recent versions of Pandoc
    \newcommand{\ConstantTok}[1]{\textcolor[rgb]{0.53,0.00,0.00}{{#1}}}
    \newcommand{\SpecialCharTok}[1]{\textcolor[rgb]{0.25,0.44,0.63}{{#1}}}
    \newcommand{\VerbatimStringTok}[1]{\textcolor[rgb]{0.25,0.44,0.63}{{#1}}}
    \newcommand{\SpecialStringTok}[1]{\textcolor[rgb]{0.73,0.40,0.53}{{#1}}}
    \newcommand{\ImportTok}[1]{{#1}}
    \newcommand{\DocumentationTok}[1]{\textcolor[rgb]{0.73,0.13,0.13}{\textit{{#1}}}}
    \newcommand{\AnnotationTok}[1]{\textcolor[rgb]{0.38,0.63,0.69}{\textbf{\textit{{#1}}}}}
    \newcommand{\CommentVarTok}[1]{\textcolor[rgb]{0.38,0.63,0.69}{\textbf{\textit{{#1}}}}}
    \newcommand{\VariableTok}[1]{\textcolor[rgb]{0.10,0.09,0.49}{{#1}}}
    \newcommand{\ControlFlowTok}[1]{\textcolor[rgb]{0.00,0.44,0.13}{\textbf{{#1}}}}
    \newcommand{\OperatorTok}[1]{\textcolor[rgb]{0.40,0.40,0.40}{{#1}}}
    \newcommand{\BuiltInTok}[1]{{#1}}
    \newcommand{\ExtensionTok}[1]{{#1}}
    \newcommand{\PreprocessorTok}[1]{\textcolor[rgb]{0.74,0.48,0.00}{{#1}}}
    \newcommand{\AttributeTok}[1]{\textcolor[rgb]{0.49,0.56,0.16}{{#1}}}
    \newcommand{\InformationTok}[1]{\textcolor[rgb]{0.38,0.63,0.69}{\textbf{\textit{{#1}}}}}
    \newcommand{\WarningTok}[1]{\textcolor[rgb]{0.38,0.63,0.69}{\textbf{\textit{{#1}}}}}
    
    
    % Define a nice break command that doesn't care if a line doesn't already
    % exist.
    \def\br{\hspace*{\fill} \\* }
    % Math Jax compatibility definitions
    \def\gt{>}
    \def\lt{<}
    \let\Oldtex\TeX
    \let\Oldlatex\LaTeX
    \renewcommand{\TeX}{\textrm{\Oldtex}}
    \renewcommand{\LaTeX}{\textrm{\Oldlatex}}
    % Document parameters
    % Document title
    \title{WebScraping\_teses\_ BDTD}
    
    
    
    
    
% Pygments definitions
\makeatletter
\def\PY@reset{\let\PY@it=\relax \let\PY@bf=\relax%
    \let\PY@ul=\relax \let\PY@tc=\relax%
    \let\PY@bc=\relax \let\PY@ff=\relax}
\def\PY@tok#1{\csname PY@tok@#1\endcsname}
\def\PY@toks#1+{\ifx\relax#1\empty\else%
    \PY@tok{#1}\expandafter\PY@toks\fi}
\def\PY@do#1{\PY@bc{\PY@tc{\PY@ul{%
    \PY@it{\PY@bf{\PY@ff{#1}}}}}}}
\def\PY#1#2{\PY@reset\PY@toks#1+\relax+\PY@do{#2}}

\expandafter\def\csname PY@tok@w\endcsname{\def\PY@tc##1{\textcolor[rgb]{0.73,0.73,0.73}{##1}}}
\expandafter\def\csname PY@tok@c\endcsname{\let\PY@it=\textit\def\PY@tc##1{\textcolor[rgb]{0.25,0.50,0.50}{##1}}}
\expandafter\def\csname PY@tok@cp\endcsname{\def\PY@tc##1{\textcolor[rgb]{0.74,0.48,0.00}{##1}}}
\expandafter\def\csname PY@tok@k\endcsname{\let\PY@bf=\textbf\def\PY@tc##1{\textcolor[rgb]{0.00,0.50,0.00}{##1}}}
\expandafter\def\csname PY@tok@kp\endcsname{\def\PY@tc##1{\textcolor[rgb]{0.00,0.50,0.00}{##1}}}
\expandafter\def\csname PY@tok@kt\endcsname{\def\PY@tc##1{\textcolor[rgb]{0.69,0.00,0.25}{##1}}}
\expandafter\def\csname PY@tok@o\endcsname{\def\PY@tc##1{\textcolor[rgb]{0.40,0.40,0.40}{##1}}}
\expandafter\def\csname PY@tok@ow\endcsname{\let\PY@bf=\textbf\def\PY@tc##1{\textcolor[rgb]{0.67,0.13,1.00}{##1}}}
\expandafter\def\csname PY@tok@nb\endcsname{\def\PY@tc##1{\textcolor[rgb]{0.00,0.50,0.00}{##1}}}
\expandafter\def\csname PY@tok@nf\endcsname{\def\PY@tc##1{\textcolor[rgb]{0.00,0.00,1.00}{##1}}}
\expandafter\def\csname PY@tok@nc\endcsname{\let\PY@bf=\textbf\def\PY@tc##1{\textcolor[rgb]{0.00,0.00,1.00}{##1}}}
\expandafter\def\csname PY@tok@nn\endcsname{\let\PY@bf=\textbf\def\PY@tc##1{\textcolor[rgb]{0.00,0.00,1.00}{##1}}}
\expandafter\def\csname PY@tok@ne\endcsname{\let\PY@bf=\textbf\def\PY@tc##1{\textcolor[rgb]{0.82,0.25,0.23}{##1}}}
\expandafter\def\csname PY@tok@nv\endcsname{\def\PY@tc##1{\textcolor[rgb]{0.10,0.09,0.49}{##1}}}
\expandafter\def\csname PY@tok@no\endcsname{\def\PY@tc##1{\textcolor[rgb]{0.53,0.00,0.00}{##1}}}
\expandafter\def\csname PY@tok@nl\endcsname{\def\PY@tc##1{\textcolor[rgb]{0.63,0.63,0.00}{##1}}}
\expandafter\def\csname PY@tok@ni\endcsname{\let\PY@bf=\textbf\def\PY@tc##1{\textcolor[rgb]{0.60,0.60,0.60}{##1}}}
\expandafter\def\csname PY@tok@na\endcsname{\def\PY@tc##1{\textcolor[rgb]{0.49,0.56,0.16}{##1}}}
\expandafter\def\csname PY@tok@nt\endcsname{\let\PY@bf=\textbf\def\PY@tc##1{\textcolor[rgb]{0.00,0.50,0.00}{##1}}}
\expandafter\def\csname PY@tok@nd\endcsname{\def\PY@tc##1{\textcolor[rgb]{0.67,0.13,1.00}{##1}}}
\expandafter\def\csname PY@tok@s\endcsname{\def\PY@tc##1{\textcolor[rgb]{0.73,0.13,0.13}{##1}}}
\expandafter\def\csname PY@tok@sd\endcsname{\let\PY@it=\textit\def\PY@tc##1{\textcolor[rgb]{0.73,0.13,0.13}{##1}}}
\expandafter\def\csname PY@tok@si\endcsname{\let\PY@bf=\textbf\def\PY@tc##1{\textcolor[rgb]{0.73,0.40,0.53}{##1}}}
\expandafter\def\csname PY@tok@se\endcsname{\let\PY@bf=\textbf\def\PY@tc##1{\textcolor[rgb]{0.73,0.40,0.13}{##1}}}
\expandafter\def\csname PY@tok@sr\endcsname{\def\PY@tc##1{\textcolor[rgb]{0.73,0.40,0.53}{##1}}}
\expandafter\def\csname PY@tok@ss\endcsname{\def\PY@tc##1{\textcolor[rgb]{0.10,0.09,0.49}{##1}}}
\expandafter\def\csname PY@tok@sx\endcsname{\def\PY@tc##1{\textcolor[rgb]{0.00,0.50,0.00}{##1}}}
\expandafter\def\csname PY@tok@m\endcsname{\def\PY@tc##1{\textcolor[rgb]{0.40,0.40,0.40}{##1}}}
\expandafter\def\csname PY@tok@gh\endcsname{\let\PY@bf=\textbf\def\PY@tc##1{\textcolor[rgb]{0.00,0.00,0.50}{##1}}}
\expandafter\def\csname PY@tok@gu\endcsname{\let\PY@bf=\textbf\def\PY@tc##1{\textcolor[rgb]{0.50,0.00,0.50}{##1}}}
\expandafter\def\csname PY@tok@gd\endcsname{\def\PY@tc##1{\textcolor[rgb]{0.63,0.00,0.00}{##1}}}
\expandafter\def\csname PY@tok@gi\endcsname{\def\PY@tc##1{\textcolor[rgb]{0.00,0.63,0.00}{##1}}}
\expandafter\def\csname PY@tok@gr\endcsname{\def\PY@tc##1{\textcolor[rgb]{1.00,0.00,0.00}{##1}}}
\expandafter\def\csname PY@tok@ge\endcsname{\let\PY@it=\textit}
\expandafter\def\csname PY@tok@gs\endcsname{\let\PY@bf=\textbf}
\expandafter\def\csname PY@tok@gp\endcsname{\let\PY@bf=\textbf\def\PY@tc##1{\textcolor[rgb]{0.00,0.00,0.50}{##1}}}
\expandafter\def\csname PY@tok@go\endcsname{\def\PY@tc##1{\textcolor[rgb]{0.53,0.53,0.53}{##1}}}
\expandafter\def\csname PY@tok@gt\endcsname{\def\PY@tc##1{\textcolor[rgb]{0.00,0.27,0.87}{##1}}}
\expandafter\def\csname PY@tok@err\endcsname{\def\PY@bc##1{\setlength{\fboxsep}{0pt}\fcolorbox[rgb]{1.00,0.00,0.00}{1,1,1}{\strut ##1}}}
\expandafter\def\csname PY@tok@kc\endcsname{\let\PY@bf=\textbf\def\PY@tc##1{\textcolor[rgb]{0.00,0.50,0.00}{##1}}}
\expandafter\def\csname PY@tok@kd\endcsname{\let\PY@bf=\textbf\def\PY@tc##1{\textcolor[rgb]{0.00,0.50,0.00}{##1}}}
\expandafter\def\csname PY@tok@kn\endcsname{\let\PY@bf=\textbf\def\PY@tc##1{\textcolor[rgb]{0.00,0.50,0.00}{##1}}}
\expandafter\def\csname PY@tok@kr\endcsname{\let\PY@bf=\textbf\def\PY@tc##1{\textcolor[rgb]{0.00,0.50,0.00}{##1}}}
\expandafter\def\csname PY@tok@bp\endcsname{\def\PY@tc##1{\textcolor[rgb]{0.00,0.50,0.00}{##1}}}
\expandafter\def\csname PY@tok@fm\endcsname{\def\PY@tc##1{\textcolor[rgb]{0.00,0.00,1.00}{##1}}}
\expandafter\def\csname PY@tok@vc\endcsname{\def\PY@tc##1{\textcolor[rgb]{0.10,0.09,0.49}{##1}}}
\expandafter\def\csname PY@tok@vg\endcsname{\def\PY@tc##1{\textcolor[rgb]{0.10,0.09,0.49}{##1}}}
\expandafter\def\csname PY@tok@vi\endcsname{\def\PY@tc##1{\textcolor[rgb]{0.10,0.09,0.49}{##1}}}
\expandafter\def\csname PY@tok@vm\endcsname{\def\PY@tc##1{\textcolor[rgb]{0.10,0.09,0.49}{##1}}}
\expandafter\def\csname PY@tok@sa\endcsname{\def\PY@tc##1{\textcolor[rgb]{0.73,0.13,0.13}{##1}}}
\expandafter\def\csname PY@tok@sb\endcsname{\def\PY@tc##1{\textcolor[rgb]{0.73,0.13,0.13}{##1}}}
\expandafter\def\csname PY@tok@sc\endcsname{\def\PY@tc##1{\textcolor[rgb]{0.73,0.13,0.13}{##1}}}
\expandafter\def\csname PY@tok@dl\endcsname{\def\PY@tc##1{\textcolor[rgb]{0.73,0.13,0.13}{##1}}}
\expandafter\def\csname PY@tok@s2\endcsname{\def\PY@tc##1{\textcolor[rgb]{0.73,0.13,0.13}{##1}}}
\expandafter\def\csname PY@tok@sh\endcsname{\def\PY@tc##1{\textcolor[rgb]{0.73,0.13,0.13}{##1}}}
\expandafter\def\csname PY@tok@s1\endcsname{\def\PY@tc##1{\textcolor[rgb]{0.73,0.13,0.13}{##1}}}
\expandafter\def\csname PY@tok@mb\endcsname{\def\PY@tc##1{\textcolor[rgb]{0.40,0.40,0.40}{##1}}}
\expandafter\def\csname PY@tok@mf\endcsname{\def\PY@tc##1{\textcolor[rgb]{0.40,0.40,0.40}{##1}}}
\expandafter\def\csname PY@tok@mh\endcsname{\def\PY@tc##1{\textcolor[rgb]{0.40,0.40,0.40}{##1}}}
\expandafter\def\csname PY@tok@mi\endcsname{\def\PY@tc##1{\textcolor[rgb]{0.40,0.40,0.40}{##1}}}
\expandafter\def\csname PY@tok@il\endcsname{\def\PY@tc##1{\textcolor[rgb]{0.40,0.40,0.40}{##1}}}
\expandafter\def\csname PY@tok@mo\endcsname{\def\PY@tc##1{\textcolor[rgb]{0.40,0.40,0.40}{##1}}}
\expandafter\def\csname PY@tok@ch\endcsname{\let\PY@it=\textit\def\PY@tc##1{\textcolor[rgb]{0.25,0.50,0.50}{##1}}}
\expandafter\def\csname PY@tok@cm\endcsname{\let\PY@it=\textit\def\PY@tc##1{\textcolor[rgb]{0.25,0.50,0.50}{##1}}}
\expandafter\def\csname PY@tok@cpf\endcsname{\let\PY@it=\textit\def\PY@tc##1{\textcolor[rgb]{0.25,0.50,0.50}{##1}}}
\expandafter\def\csname PY@tok@c1\endcsname{\let\PY@it=\textit\def\PY@tc##1{\textcolor[rgb]{0.25,0.50,0.50}{##1}}}
\expandafter\def\csname PY@tok@cs\endcsname{\let\PY@it=\textit\def\PY@tc##1{\textcolor[rgb]{0.25,0.50,0.50}{##1}}}

\def\PYZbs{\char`\\}
\def\PYZus{\char`\_}
\def\PYZob{\char`\{}
\def\PYZcb{\char`\}}
\def\PYZca{\char`\^}
\def\PYZam{\char`\&}
\def\PYZlt{\char`\<}
\def\PYZgt{\char`\>}
\def\PYZsh{\char`\#}
\def\PYZpc{\char`\%}
\def\PYZdl{\char`\$}
\def\PYZhy{\char`\-}
\def\PYZsq{\char`\'}
\def\PYZdq{\char`\"}
\def\PYZti{\char`\~}
% for compatibility with earlier versions
\def\PYZat{@}
\def\PYZlb{[}
\def\PYZrb{]}
\makeatother


    % For linebreaks inside Verbatim environment from package fancyvrb. 
    \makeatletter
        \newbox\Wrappedcontinuationbox 
        \newbox\Wrappedvisiblespacebox 
        \newcommand*\Wrappedvisiblespace {\textcolor{red}{\textvisiblespace}} 
        \newcommand*\Wrappedcontinuationsymbol {\textcolor{red}{\llap{\tiny$\m@th\hookrightarrow$}}} 
        \newcommand*\Wrappedcontinuationindent {3ex } 
        \newcommand*\Wrappedafterbreak {\kern\Wrappedcontinuationindent\copy\Wrappedcontinuationbox} 
        % Take advantage of the already applied Pygments mark-up to insert 
        % potential linebreaks for TeX processing. 
        %        {, <, #, %, $, ' and ": go to next line. 
        %        _, }, ^, &, >, - and ~: stay at end of broken line. 
        % Use of \textquotesingle for straight quote. 
        \newcommand*\Wrappedbreaksatspecials {% 
            \def\PYGZus{\discretionary{\char`\_}{\Wrappedafterbreak}{\char`\_}}% 
            \def\PYGZob{\discretionary{}{\Wrappedafterbreak\char`\{}{\char`\{}}% 
            \def\PYGZcb{\discretionary{\char`\}}{\Wrappedafterbreak}{\char`\}}}% 
            \def\PYGZca{\discretionary{\char`\^}{\Wrappedafterbreak}{\char`\^}}% 
            \def\PYGZam{\discretionary{\char`\&}{\Wrappedafterbreak}{\char`\&}}% 
            \def\PYGZlt{\discretionary{}{\Wrappedafterbreak\char`\<}{\char`\<}}% 
            \def\PYGZgt{\discretionary{\char`\>}{\Wrappedafterbreak}{\char`\>}}% 
            \def\PYGZsh{\discretionary{}{\Wrappedafterbreak\char`\#}{\char`\#}}% 
            \def\PYGZpc{\discretionary{}{\Wrappedafterbreak\char`\%}{\char`\%}}% 
            \def\PYGZdl{\discretionary{}{\Wrappedafterbreak\char`\$}{\char`\$}}% 
            \def\PYGZhy{\discretionary{\char`\-}{\Wrappedafterbreak}{\char`\-}}% 
            \def\PYGZsq{\discretionary{}{\Wrappedafterbreak\textquotesingle}{\textquotesingle}}% 
            \def\PYGZdq{\discretionary{}{\Wrappedafterbreak\char`\"}{\char`\"}}% 
            \def\PYGZti{\discretionary{\char`\~}{\Wrappedafterbreak}{\char`\~}}% 
        } 
        % Some characters . , ; ? ! / are not pygmentized. 
        % This macro makes them "active" and they will insert potential linebreaks 
        \newcommand*\Wrappedbreaksatpunct {% 
            \lccode`\~`\.\lowercase{\def~}{\discretionary{\hbox{\char`\.}}{\Wrappedafterbreak}{\hbox{\char`\.}}}% 
            \lccode`\~`\,\lowercase{\def~}{\discretionary{\hbox{\char`\,}}{\Wrappedafterbreak}{\hbox{\char`\,}}}% 
            \lccode`\~`\;\lowercase{\def~}{\discretionary{\hbox{\char`\;}}{\Wrappedafterbreak}{\hbox{\char`\;}}}% 
            \lccode`\~`\:\lowercase{\def~}{\discretionary{\hbox{\char`\:}}{\Wrappedafterbreak}{\hbox{\char`\:}}}% 
            \lccode`\~`\?\lowercase{\def~}{\discretionary{\hbox{\char`\?}}{\Wrappedafterbreak}{\hbox{\char`\?}}}% 
            \lccode`\~`\!\lowercase{\def~}{\discretionary{\hbox{\char`\!}}{\Wrappedafterbreak}{\hbox{\char`\!}}}% 
            \lccode`\~`\/\lowercase{\def~}{\discretionary{\hbox{\char`\/}}{\Wrappedafterbreak}{\hbox{\char`\/}}}% 
            \catcode`\.\active
            \catcode`\,\active 
            \catcode`\;\active
            \catcode`\:\active
            \catcode`\?\active
            \catcode`\!\active
            \catcode`\/\active 
            \lccode`\~`\~ 	
        }
    \makeatother

    \let\OriginalVerbatim=\Verbatim
    \makeatletter
    \renewcommand{\Verbatim}[1][1]{%
        %\parskip\z@skip
        \sbox\Wrappedcontinuationbox {\Wrappedcontinuationsymbol}%
        \sbox\Wrappedvisiblespacebox {\FV@SetupFont\Wrappedvisiblespace}%
        \def\FancyVerbFormatLine ##1{\hsize\linewidth
            \vtop{\raggedright\hyphenpenalty\z@\exhyphenpenalty\z@
                \doublehyphendemerits\z@\finalhyphendemerits\z@
                \strut ##1\strut}%
        }%
        % If the linebreak is at a space, the latter will be displayed as visible
        % space at end of first line, and a continuation symbol starts next line.
        % Stretch/shrink are however usually zero for typewriter font.
        \def\FV@Space {%
            \nobreak\hskip\z@ plus\fontdimen3\font minus\fontdimen4\font
            \discretionary{\copy\Wrappedvisiblespacebox}{\Wrappedafterbreak}
            {\kern\fontdimen2\font}%
        }%
        
        % Allow breaks at special characters using \PYG... macros.
        \Wrappedbreaksatspecials
        % Breaks at punctuation characters . , ; ? ! and / need catcode=\active 	
        \OriginalVerbatim[#1,codes*=\Wrappedbreaksatpunct]%
    }
    \makeatother

    % Exact colors from NB
    \definecolor{incolor}{HTML}{303F9F}
    \definecolor{outcolor}{HTML}{D84315}
    \definecolor{cellborder}{HTML}{CFCFCF}
    \definecolor{cellbackground}{HTML}{F7F7F7}
    
    % prompt
    \newcommand{\prompt}[4]{
        \llap{{\color{#2}[#3]: #4}}\vspace{-1.25em}
    }
    

    
    % Prevent overflowing lines due to hard-to-break entities
    \sloppy 
    % Setup hyperref package
    \hypersetup{
      breaklinks=true,  % so long urls are correctly broken across lines
      colorlinks=true,
      urlcolor=urlcolor,
      linkcolor=linkcolor,
      citecolor=citecolor,
      }
    % Slightly bigger margins than the latex defaults
    
    \geometry{verbose,tmargin=1in,bmargin=1in,lmargin=1in,rmargin=1in}
    
    

    \begin{document}
    
    
    \maketitle
    
    

    
    Script para buscar resumos na BDTD, testar se eles são relevantes para o
domínio de óleo e gás e baixar o documento original no repositório
institucional.

    \begin{tcolorbox}[breakable, size=fbox, boxrule=1pt, pad at break*=1mm,colback=cellbackground, colframe=cellborder]
\prompt{In}{incolor}{1}{\hspace{4pt}}
\begin{Verbatim}[commandchars=\\\{\}]
\PY{k+kn}{import} \PY{n+nn}{requests}
\PY{k+kn}{from} \PY{n+nn}{bs4} \PY{k}{import} \PY{n}{BeautifulSoup} \PY{k}{as} \PY{n}{bs}
\PY{k+kn}{import} \PY{n+nn}{pandas} \PY{k}{as} \PY{n+nn}{pd}
\PY{k+kn}{import} \PY{n+nn}{numpy} \PY{k}{as} \PY{n+nn}{np}
\PY{k+kn}{import} \PY{n+nn}{json}
\PY{k+kn}{import} \PY{n+nn}{nltk}
\PY{k+kn}{from} \PY{n+nn}{nltk}\PY{n+nn}{.}\PY{n+nn}{tokenize} \PY{k}{import} \PY{n}{word\PYZus{}tokenize}
\PY{k+kn}{from} \PY{n+nn}{langdetect} \PY{k}{import} \PY{n}{detect}
\PY{k+kn}{from} \PY{n+nn}{langdetect} \PY{k}{import} \PY{n}{detect\PYZus{}langs}
\PY{k+kn}{from} \PY{n+nn}{keras}\PY{n+nn}{.}\PY{n+nn}{models} \PY{k}{import} \PY{n}{load\PYZus{}model}
\PY{k+kn}{import} \PY{n+nn}{gensim}
\PY{k+kn}{from} \PY{n+nn}{gensim}\PY{n+nn}{.}\PY{n+nn}{models} \PY{k}{import} \PY{n}{Word2Vec}
\PY{k+kn}{import} \PY{n+nn}{csv}
\PY{k+kn}{import} \PY{n+nn}{re}
\end{Verbatim}
\end{tcolorbox}

    \begin{Verbatim}[commandchars=\\\{\}]
Using TensorFlow backend.
C:\textbackslash{}ProgramData\textbackslash{}Anaconda3\textbackslash{}lib\textbackslash{}site-
packages\textbackslash{}tensorflow\textbackslash{}python\textbackslash{}framework\textbackslash{}dtypes.py:516: FutureWarning: Passing
(type, 1) or '1type' as a synonym of type is deprecated; in a future version of
numpy, it will be understood as (type, (1,)) / '(1,)type'.
  \_np\_qint8 = np.dtype([("qint8", np.int8, 1)])
C:\textbackslash{}ProgramData\textbackslash{}Anaconda3\textbackslash{}lib\textbackslash{}site-
packages\textbackslash{}tensorflow\textbackslash{}python\textbackslash{}framework\textbackslash{}dtypes.py:517: FutureWarning: Passing
(type, 1) or '1type' as a synonym of type is deprecated; in a future version of
numpy, it will be understood as (type, (1,)) / '(1,)type'.
  \_np\_quint8 = np.dtype([("quint8", np.uint8, 1)])
C:\textbackslash{}ProgramData\textbackslash{}Anaconda3\textbackslash{}lib\textbackslash{}site-
packages\textbackslash{}tensorflow\textbackslash{}python\textbackslash{}framework\textbackslash{}dtypes.py:518: FutureWarning: Passing
(type, 1) or '1type' as a synonym of type is deprecated; in a future version of
numpy, it will be understood as (type, (1,)) / '(1,)type'.
  \_np\_qint16 = np.dtype([("qint16", np.int16, 1)])
C:\textbackslash{}ProgramData\textbackslash{}Anaconda3\textbackslash{}lib\textbackslash{}site-
packages\textbackslash{}tensorflow\textbackslash{}python\textbackslash{}framework\textbackslash{}dtypes.py:519: FutureWarning: Passing
(type, 1) or '1type' as a synonym of type is deprecated; in a future version of
numpy, it will be understood as (type, (1,)) / '(1,)type'.
  \_np\_quint16 = np.dtype([("quint16", np.uint16, 1)])
C:\textbackslash{}ProgramData\textbackslash{}Anaconda3\textbackslash{}lib\textbackslash{}site-
packages\textbackslash{}tensorflow\textbackslash{}python\textbackslash{}framework\textbackslash{}dtypes.py:520: FutureWarning: Passing
(type, 1) or '1type' as a synonym of type is deprecated; in a future version of
numpy, it will be understood as (type, (1,)) / '(1,)type'.
  \_np\_qint32 = np.dtype([("qint32", np.int32, 1)])
C:\textbackslash{}ProgramData\textbackslash{}Anaconda3\textbackslash{}lib\textbackslash{}site-
packages\textbackslash{}tensorflow\textbackslash{}python\textbackslash{}framework\textbackslash{}dtypes.py:525: FutureWarning: Passing
(type, 1) or '1type' as a synonym of type is deprecated; in a future version of
numpy, it will be understood as (type, (1,)) / '(1,)type'.
  np\_resource = np.dtype([("resource", np.ubyte, 1)])
C:\textbackslash{}Users\textbackslash{}upe2\textbackslash{}AppData\textbackslash{}Roaming\textbackslash{}Python\textbackslash{}Python36\textbackslash{}site-
packages\textbackslash{}tensorboard\textbackslash{}compat\textbackslash{}tensorflow\_stub\textbackslash{}dtypes.py:541: FutureWarning:
Passing (type, 1) or '1type' as a synonym of type is deprecated; in a future
version of numpy, it will be understood as (type, (1,)) / '(1,)type'.
  \_np\_qint8 = np.dtype([("qint8", np.int8, 1)])
C:\textbackslash{}Users\textbackslash{}upe2\textbackslash{}AppData\textbackslash{}Roaming\textbackslash{}Python\textbackslash{}Python36\textbackslash{}site-
packages\textbackslash{}tensorboard\textbackslash{}compat\textbackslash{}tensorflow\_stub\textbackslash{}dtypes.py:542: FutureWarning:
Passing (type, 1) or '1type' as a synonym of type is deprecated; in a future
version of numpy, it will be understood as (type, (1,)) / '(1,)type'.
  \_np\_quint8 = np.dtype([("quint8", np.uint8, 1)])
C:\textbackslash{}Users\textbackslash{}upe2\textbackslash{}AppData\textbackslash{}Roaming\textbackslash{}Python\textbackslash{}Python36\textbackslash{}site-
packages\textbackslash{}tensorboard\textbackslash{}compat\textbackslash{}tensorflow\_stub\textbackslash{}dtypes.py:543: FutureWarning:
Passing (type, 1) or '1type' as a synonym of type is deprecated; in a future
version of numpy, it will be understood as (type, (1,)) / '(1,)type'.
  \_np\_qint16 = np.dtype([("qint16", np.int16, 1)])
C:\textbackslash{}Users\textbackslash{}upe2\textbackslash{}AppData\textbackslash{}Roaming\textbackslash{}Python\textbackslash{}Python36\textbackslash{}site-
packages\textbackslash{}tensorboard\textbackslash{}compat\textbackslash{}tensorflow\_stub\textbackslash{}dtypes.py:544: FutureWarning:
Passing (type, 1) or '1type' as a synonym of type is deprecated; in a future
version of numpy, it will be understood as (type, (1,)) / '(1,)type'.
  \_np\_quint16 = np.dtype([("quint16", np.uint16, 1)])
C:\textbackslash{}Users\textbackslash{}upe2\textbackslash{}AppData\textbackslash{}Roaming\textbackslash{}Python\textbackslash{}Python36\textbackslash{}site-
packages\textbackslash{}tensorboard\textbackslash{}compat\textbackslash{}tensorflow\_stub\textbackslash{}dtypes.py:545: FutureWarning:
Passing (type, 1) or '1type' as a synonym of type is deprecated; in a future
version of numpy, it will be understood as (type, (1,)) / '(1,)type'.
  \_np\_qint32 = np.dtype([("qint32", np.int32, 1)])
C:\textbackslash{}Users\textbackslash{}upe2\textbackslash{}AppData\textbackslash{}Roaming\textbackslash{}Python\textbackslash{}Python36\textbackslash{}site-
packages\textbackslash{}tensorboard\textbackslash{}compat\textbackslash{}tensorflow\_stub\textbackslash{}dtypes.py:550: FutureWarning:
Passing (type, 1) or '1type' as a synonym of type is deprecated; in a future
version of numpy, it will be understood as (type, (1,)) / '(1,)type'.
  np\_resource = np.dtype([("resource", np.ubyte, 1)])
C:\textbackslash{}ProgramData\textbackslash{}Anaconda3\textbackslash{}lib\textbackslash{}site-packages\textbackslash{}gensim\textbackslash{}utils.py:1197: UserWarning:
detected Windows; aliasing chunkize to chunkize\_serial
  warnings.warn("detected Windows; aliasing chunkize to chunkize\_serial")
\end{Verbatim}

    \begin{tcolorbox}[breakable, size=fbox, boxrule=1pt, pad at break*=1mm,colback=cellbackground, colframe=cellborder]
\prompt{In}{incolor}{2}{\hspace{4pt}}
\begin{Verbatim}[commandchars=\\\{\}]
\PY{c+c1}{\PYZsh{} Definindo configurações globais de proxy para realizar a extração dentro da rede Petrobras}
\PY{n}{chave} \PY{o}{=} \PY{l+s+s1}{\PYZsq{}}\PY{l+s+s1}{XXXX}\PY{l+s+s1}{\PYZsq{}}
\PY{n}{pwd} \PY{o}{=} \PY{l+s+s1}{\PYZsq{}}\PY{l+s+s1}{XXXXXXXXXX}\PY{l+s+s1}{\PYZsq{}}
\PY{n}{proxy\PYZus{}url} \PY{o}{=} \PY{l+s+s1}{\PYZsq{}}\PY{l+s+s1}{http://}\PY{l+s+s1}{\PYZsq{}}\PY{o}{+}\PY{n}{chave}\PY{o}{+}\PY{l+s+s1}{\PYZsq{}}\PY{l+s+s1}{:}\PY{l+s+s1}{\PYZsq{}}\PY{o}{+}\PY{n}{pwd}\PY{o}{+}\PY{l+s+s1}{\PYZsq{}}\PY{l+s+s1}{@inet\PYZhy{}sys.gnet.petrobras.com.br:804/}\PY{l+s+s1}{\PYZsq{}}
\PY{n}{proxies} \PY{o}{=} \PY{p}{\PYZob{}}
  \PY{l+s+s1}{\PYZsq{}}\PY{l+s+s1}{http}\PY{l+s+s1}{\PYZsq{}} \PY{p}{:} \PY{n}{proxy\PYZus{}url} \PY{p}{,}
  \PY{l+s+s1}{\PYZsq{}}\PY{l+s+s1}{https}\PY{l+s+s1}{\PYZsq{}} \PY{p}{:} \PY{n}{proxy\PYZus{}url} \PY{p}{,}
\PY{p}{\PYZcb{}}
\end{Verbatim}
\end{tcolorbox}

    Inicialmente entraremos no site da BDTD e buscaremos os links de todas
as teses de uma determinada intituição.

    \begin{tcolorbox}[breakable, size=fbox, boxrule=1pt, pad at break*=1mm,colback=cellbackground, colframe=cellborder]
\prompt{In}{incolor}{3}{\hspace{4pt}}
\begin{Verbatim}[commandchars=\\\{\}]
\PY{c+c1}{\PYZsh{}função para coletar os links das tese}

\PY{k}{def} \PY{n+nf}{get\PYZus{}links}\PY{p}{(}\PY{n}{page}\PY{p}{)}\PY{p}{:}
        
    \PY{c+c1}{\PYZsh{}preparar a url}
    \PY{n}{url} \PY{o}{=} \PY{p}{(}\PY{l+s+s1}{\PYZsq{}}\PY{l+s+s1}{http://bdtd.ibict.br/vufind/Search/Results?filter}\PY{l+s+s1}{\PYZpc{}}\PY{l+s+s1}{5B}\PY{l+s+s1}{\PYZpc{}}\PY{l+s+s1}{5D=institution}\PY{l+s+s1}{\PYZpc{}}\PY{l+s+s1}{3A}\PY{l+s+s1}{\PYZpc{}}\PY{l+s+s1}{22UFBA}\PY{l+s+s1}{\PYZpc{}}\PY{l+s+s1}{22\PYZam{}type=AllFields\PYZam{}page=}\PY{l+s+s1}{\PYZsq{}} \PY{o}{+}
           \PY{n+nb}{str}\PY{p}{(}\PY{n}{page}\PY{p}{)}\PY{p}{)}
    
    \PY{c+c1}{\PYZsh{}Fazer requisição e parsear o arquivo html}
    \PY{n}{f} \PY{o}{=} \PY{n}{requests}\PY{o}{.}\PY{n}{get}\PY{p}{(}\PY{n}{url}\PY{p}{,} \PY{n}{proxies} \PY{o}{=} \PY{n}{proxies}\PY{p}{)}\PY{o}{.}\PY{n}{text} 
    \PY{n}{soup} \PY{o}{=} \PY{n}{bs}\PY{p}{(}\PY{n}{f}\PY{p}{,} \PY{l+s+s2}{\PYZdq{}}\PY{l+s+s2}{html.parser}\PY{l+s+s2}{\PYZdq{}}\PY{p}{)}

    \PY{c+c1}{\PYZsh{}Coletando link para as teses}
    \PY{n}{links} \PY{o}{=} \PY{p}{[}\PY{p}{]}
    \PY{k}{for} \PY{n}{doc} \PY{o+ow}{in} \PY{n}{soup}\PY{o}{.}\PY{n}{find\PYZus{}all}\PY{p}{(}\PY{l+s+s1}{\PYZsq{}}\PY{l+s+s1}{a}\PY{l+s+s1}{\PYZsq{}}\PY{p}{,} \PY{n}{href}\PY{o}{=}\PY{k+kc}{True}\PY{p}{)}\PY{p}{:}
        \PY{k}{if} \PY{l+s+s1}{\PYZsq{}}\PY{l+s+s1}{title}\PY{l+s+s1}{\PYZsq{}} \PY{o+ow}{in} \PY{n}{doc}\PY{o}{.}\PY{n}{get}\PY{p}{(}\PY{l+s+s1}{\PYZsq{}}\PY{l+s+s1}{class}\PY{l+s+s1}{\PYZsq{}}\PY{p}{,} \PY{p}{[}\PY{p}{]}\PY{p}{)}\PY{p}{:}
            \PY{n}{links}\PY{o}{.}\PY{n}{append}\PY{p}{(}\PY{n}{doc}\PY{p}{[}\PY{l+s+s1}{\PYZsq{}}\PY{l+s+s1}{href}\PY{l+s+s1}{\PYZsq{}}\PY{p}{]}\PY{p}{)}
    \PY{k}{return} \PY{n}{links}
\end{Verbatim}
\end{tcolorbox}

    \begin{tcolorbox}[breakable, size=fbox, boxrule=1pt, pad at break*=1mm,colback=cellbackground, colframe=cellborder]
\prompt{In}{incolor}{4}{\hspace{4pt}}
\begin{Verbatim}[commandchars=\\\{\}]
\PY{c+c1}{\PYZsh{}Coletar o link de todas as teses}
\PY{n}{start\PYZus{}page} \PY{o}{=} \PY{l+m+mi}{1}
\PY{n}{n\PYZus{}pages} \PY{o}{=} \PY{l+m+mi}{500} \PY{c+c1}{\PYZsh{} Cada página retorna 20 teses}

\PY{n}{links} \PY{o}{=} \PY{p}{[}\PY{p}{]}

\PY{k}{for} \PY{n}{p} \PY{o+ow}{in} \PY{n+nb}{range}\PY{p}{(}\PY{n}{start\PYZus{}page}\PY{p}{,} \PY{n}{n\PYZus{}pages}\PY{p}{)}\PY{p}{:}
    \PY{n}{link} \PY{o}{=} \PY{n}{get\PYZus{}links}\PY{p}{(}\PY{n}{p}\PY{p}{)}
    \PY{k}{if} \PY{n}{link} \PY{o}{!=} \PY{p}{[}\PY{p}{]}\PY{p}{:}
        \PY{n}{links} \PY{o}{=} \PY{n}{links} \PY{o}{+} \PY{n}{link}
    \PY{k}{else}\PY{p}{:}
        \PY{k}{break}
    
    \PY{k}{if} \PY{n}{p} \PY{o}{\PYZpc{}} \PY{l+m+mi}{100} \PY{o}{==} \PY{l+m+mi}{0}\PY{p}{:}
        \PY{n+nb}{print} \PY{p}{(}\PY{n}{p}\PY{o}{*}\PY{l+m+mi}{20}\PY{p}{,} \PY{l+s+s1}{\PYZsq{}}\PY{l+s+s1}{ links capturados, }\PY{l+s+s1}{\PYZsq{}}\PY{p}{,} \PY{n}{p}\PY{p}{,} \PY{l+s+s1}{\PYZsq{}}\PY{l+s+s1}{ páginas}\PY{l+s+s1}{\PYZsq{}}\PY{p}{)}
        \PY{k}{with} \PY{n+nb}{open}\PY{p}{(}\PY{l+s+s1}{\PYZsq{}}\PY{l+s+s1}{links\PYZus{}ufba}\PY{l+s+s1}{\PYZsq{}}\PY{p}{,} \PY{l+s+s2}{\PYZdq{}}\PY{l+s+s2}{w}\PY{l+s+s2}{\PYZdq{}}\PY{p}{)} \PY{k}{as} \PY{n}{output}\PY{p}{:}
            \PY{n}{writer} \PY{o}{=} \PY{n}{csv}\PY{o}{.}\PY{n}{writer}\PY{p}{(}\PY{n}{output}\PY{p}{,} \PY{n}{lineterminator}\PY{o}{=}\PY{l+s+s1}{\PYZsq{}}\PY{l+s+se}{\PYZbs{}n}\PY{l+s+s1}{\PYZsq{}}\PY{p}{)}
            \PY{k}{for} \PY{n}{val} \PY{o+ow}{in} \PY{n}{links}\PY{p}{:}
                \PY{n}{writer}\PY{o}{.}\PY{n}{writerow}\PY{p}{(}\PY{p}{[}\PY{n}{val}\PY{p}{]}\PY{p}{)}
                
\PY{k}{with} \PY{n+nb}{open}\PY{p}{(}\PY{l+s+s1}{\PYZsq{}}\PY{l+s+s1}{links\PYZus{}ufba}\PY{l+s+s1}{\PYZsq{}}\PY{p}{,} \PY{l+s+s2}{\PYZdq{}}\PY{l+s+s2}{w}\PY{l+s+s2}{\PYZdq{}}\PY{p}{)} \PY{k}{as} \PY{n}{output}\PY{p}{:}
    \PY{n}{writer} \PY{o}{=} \PY{n}{csv}\PY{o}{.}\PY{n}{writer}\PY{p}{(}\PY{n}{output}\PY{p}{,} \PY{n}{lineterminator}\PY{o}{=}\PY{l+s+s1}{\PYZsq{}}\PY{l+s+se}{\PYZbs{}n}\PY{l+s+s1}{\PYZsq{}}\PY{p}{)}
    \PY{k}{for} \PY{n}{val} \PY{o+ow}{in} \PY{n}{links}\PY{p}{:}
        \PY{n}{writer}\PY{o}{.}\PY{n}{writerow}\PY{p}{(}\PY{p}{[}\PY{n}{val}\PY{p}{]}\PY{p}{)} 
\PY{n+nb}{print} \PY{p}{(}\PY{n}{p}\PY{o}{*}\PY{l+m+mi}{20}\PY{p}{,} \PY{l+s+s1}{\PYZsq{}}\PY{l+s+s1}{ links capturados, }\PY{l+s+s1}{\PYZsq{}}\PY{p}{,} \PY{n}{p}\PY{p}{,} \PY{l+s+s1}{\PYZsq{}}\PY{l+s+s1}{ páginas}\PY{l+s+s1}{\PYZsq{}}\PY{p}{)}
\end{Verbatim}
\end{tcolorbox}

    \begin{Verbatim}[commandchars=\\\{\}]
2000  links capturados,  100  páginas
4000  links capturados,  200  páginas
6000  links capturados,  300  páginas
8000  links capturados,  400  páginas
9940  links capturados,  497  páginas
\end{Verbatim}

    \begin{tcolorbox}[breakable, size=fbox, boxrule=1pt, pad at break*=1mm,colback=cellbackground, colframe=cellborder]
\prompt{In}{incolor}{5}{\hspace{4pt}}
\begin{Verbatim}[commandchars=\\\{\}]
\PY{c+c1}{\PYZsh{} Abrindo arquivo gravado anteriormente}

\PY{c+c1}{\PYZsh{}links = []}
\PY{c+c1}{\PYZsh{}with open(\PYZsq{}links\PYZus{}ufba\PYZsq{}, \PYZsq{}r\PYZsq{}) as f:}
\PY{c+c1}{\PYZsh{}    reader = csv.reader(f)}
\PY{c+c1}{\PYZsh{}    for link in reader:}
\PY{c+c1}{\PYZsh{}        links.append(link[0])}
\end{Verbatim}
\end{tcolorbox}

    Em seguida vamos recuperar os metadados de cada link coletado
anteriormente.

    \begin{tcolorbox}[breakable, size=fbox, boxrule=1pt, pad at break*=1mm,colback=cellbackground, colframe=cellborder]
\prompt{In}{incolor}{6}{\hspace{4pt}}
\begin{Verbatim}[commandchars=\\\{\}]
\PY{c+c1}{\PYZsh{}função para buscar os metadados das teses no BDTD}
\PY{k}{def} \PY{n+nf}{tese\PYZus{}link}\PY{p}{(}\PY{n}{link}\PY{p}{)}\PY{p}{:}
    \PY{c+c1}{\PYZsh{}definir url}
    \PY{n}{url} \PY{o}{=} \PY{l+s+s1}{\PYZsq{}}\PY{l+s+s1}{http://bdtd.ibict.br}\PY{l+s+s1}{\PYZsq{}} \PY{o}{+} \PY{n}{link}
    
    \PY{c+c1}{\PYZsh{}Requisitar html e fazer o parser}
    \PY{n}{f} \PY{o}{=} \PY{n}{requests}\PY{o}{.}\PY{n}{get}\PY{p}{(}\PY{n}{url}\PY{p}{,} \PY{n}{proxies} \PY{o}{=} \PY{n}{proxies}\PY{p}{)}\PY{o}{.}\PY{n}{text} 
    \PY{n}{soup} \PY{o}{=} \PY{n}{bs}\PY{p}{(}\PY{n}{f}\PY{p}{,} \PY{l+s+s2}{\PYZdq{}}\PY{l+s+s2}{html.parser}\PY{l+s+s2}{\PYZdq{}}\PY{p}{)}

    \PY{c+c1}{\PYZsh{}Dicionário para armazenar as informações da tese}
    \PY{n}{tese} \PY{o}{=} \PY{p}{\PYZob{}}\PY{p}{\PYZcb{}}  
    
    \PY{c+c1}{\PYZsh{}Adicionar título}
    \PY{n}{tese}\PY{p}{[}\PY{l+s+s1}{\PYZsq{}}\PY{l+s+s1}{Title}\PY{l+s+s1}{\PYZsq{}}\PY{p}{]} \PY{o}{=} \PY{n}{soup}\PY{o}{.}\PY{n}{find}\PY{p}{(}\PY{l+s+s1}{\PYZsq{}}\PY{l+s+s1}{h3}\PY{l+s+s1}{\PYZsq{}}\PY{p}{)}\PY{o}{.}\PY{n}{get\PYZus{}text}\PY{p}{(}\PY{p}{)}
    \PY{k}{for} \PY{n}{doc} \PY{o+ow}{in} \PY{n}{soup}\PY{o}{.}\PY{n}{find\PYZus{}all}\PY{p}{(}\PY{l+s+s1}{\PYZsq{}}\PY{l+s+s1}{tr}\PY{l+s+s1}{\PYZsq{}}\PY{p}{)}\PY{p}{:}
        \PY{c+c1}{\PYZsh{}Identificar atributo}
        \PY{k}{try}\PY{p}{:}
            \PY{n}{atributo} \PY{o}{=} \PY{n}{doc}\PY{o}{.}\PY{n}{find}\PY{p}{(}\PY{l+s+s1}{\PYZsq{}}\PY{l+s+s1}{th}\PY{l+s+s1}{\PYZsq{}}\PY{p}{)}\PY{o}{.}\PY{n}{get\PYZus{}text}\PY{p}{(}\PY{p}{)}
        \PY{k}{except}\PY{p}{:}
            \PY{k}{pass}
        \PY{c+c1}{\PYZsh{}Verificar se o atributo possui mais de um dado}
        \PY{k}{for} \PY{n}{row} \PY{o+ow}{in} \PY{n}{doc}\PY{o}{.}\PY{n}{find\PYZus{}all}\PY{p}{(}\PY{l+s+s1}{\PYZsq{}}\PY{l+s+s1}{td}\PY{l+s+s1}{\PYZsq{}}\PY{p}{)}\PY{p}{:}
            \PY{c+c1}{\PYZsh{}Adicionar o atributo no dicionário}
            \PY{k}{if} \PY{n}{row}\PY{o}{.}\PY{n}{find}\PY{p}{(}\PY{l+s+s1}{\PYZsq{}}\PY{l+s+s1}{div}\PY{l+s+s1}{\PYZsq{}}\PY{p}{)} \PY{o}{==} \PY{k+kc}{None}\PY{p}{:}
                \PY{k}{try}\PY{p}{:}
                    \PY{n}{tese}\PY{p}{[}\PY{n}{atributo}\PY{p}{]} \PY{o}{=} \PY{n}{doc}\PY{o}{.}\PY{n}{find}\PY{p}{(}\PY{l+s+s1}{\PYZsq{}}\PY{l+s+s1}{td}\PY{l+s+s1}{\PYZsq{}}\PY{p}{)}\PY{o}{.}\PY{n}{get\PYZus{}text}\PY{p}{(}\PY{p}{)}
                \PY{k}{except}\PY{p}{:}
                    \PY{k}{pass}
            \PY{k}{else}\PY{p}{:}
                \PY{n}{element} \PY{o}{=} \PY{p}{[}\PY{p}{]}
                \PY{c+c1}{\PYZsh{}No dicionário, adicionar todos os dados ao seu respectivo atributo}
                \PY{k}{for} \PY{n}{e} \PY{o+ow}{in} \PY{n}{doc}\PY{o}{.}\PY{n}{find\PYZus{}all}\PY{p}{(}\PY{l+s+s1}{\PYZsq{}}\PY{l+s+s1}{div}\PY{l+s+s1}{\PYZsq{}}\PY{p}{)}\PY{p}{:}
                    \PY{k}{try}\PY{p}{:}
                        \PY{n}{sub\PYZus{}e} \PY{o}{=} \PY{p}{[}\PY{p}{]}
                        \PY{k}{for} \PY{n}{sub\PYZus{}element} \PY{o+ow}{in} \PY{n}{e}\PY{o}{.}\PY{n}{find\PYZus{}all}\PY{p}{(}\PY{l+s+s1}{\PYZsq{}}\PY{l+s+s1}{a}\PY{l+s+s1}{\PYZsq{}}\PY{p}{)}\PY{p}{:}
                            \PY{n}{element}\PY{o}{.}\PY{n}{append}\PY{p}{(}\PY{n}{sub\PYZus{}element}\PY{o}{.}\PY{n}{get\PYZus{}text}\PY{p}{(}\PY{p}{)}\PY{p}{)} 
                        \PY{c+c1}{\PYZsh{}element.append(sub\PYZus{}e)}
                    \PY{k}{except}\PY{p}{:}
                        \PY{k}{pass}
                \PY{n}{tese}\PY{p}{[}\PY{n}{atributo}\PY{p}{]} \PY{o}{=} \PY{n}{element}
    
    \PY{k}{return}\PY{p}{(}\PY{n}{tese}\PY{p}{)}
\end{Verbatim}
\end{tcolorbox}

    Como em alguns casos o resumo português e inglês se misturaram, foi
implementado uma função para separar os textos misturados

    \begin{tcolorbox}[breakable, size=fbox, boxrule=1pt, pad at break*=1mm,colback=cellbackground, colframe=cellborder]
\prompt{In}{incolor}{7}{\hspace{4pt}}
\begin{Verbatim}[commandchars=\\\{\}]
\PY{c+c1}{\PYZsh{} Função para separar resumos português e inglês}
\PY{k}{def} \PY{n+nf}{separacao\PYZus{}port\PYZus{}engl}\PY{p}{(}\PY{n}{abstract}\PY{p}{)}\PY{p}{:}

    \PY{n}{mix\PYZus{}sent} \PY{o}{=} \PY{n}{nltk}\PY{o}{.}\PY{n}{sent\PYZus{}tokenize}\PY{p}{(}\PY{n}{abstract}\PY{p}{)}

    \PY{n}{new\PYZus{}mix} \PY{o}{=} \PY{p}{[}\PY{p}{]}
    \PY{k}{for} \PY{n}{sent} \PY{o+ow}{in} \PY{n}{mix\PYZus{}sent}\PY{p}{:}
        \PY{n}{position} \PY{o}{=} \PY{n}{sent}\PY{o}{.}\PY{n}{find}\PY{p}{(}\PY{l+s+s1}{\PYZsq{}}\PY{l+s+s1}{.}\PY{l+s+s1}{\PYZsq{}}\PY{p}{)}
        \PY{k}{if} \PY{n}{position} \PY{o}{!=} \PY{n+nb}{len}\PY{p}{(}\PY{n}{sent}\PY{p}{)}\PY{o}{\PYZhy{}}\PY{l+m+mi}{1}\PY{p}{:}
            \PY{n}{sent\PYZus{}1} \PY{o}{=} \PY{n}{sent}\PY{p}{[}\PY{p}{:}\PY{n}{position}\PY{o}{+}\PY{l+m+mi}{1}\PY{p}{]}
            \PY{n}{sent\PYZus{}2} \PY{o}{=} \PY{n}{sent}\PY{p}{[}\PY{n}{position}\PY{o}{+}\PY{l+m+mi}{1}\PY{p}{:}\PY{p}{]}
            \PY{n}{new\PYZus{}mix}\PY{o}{.}\PY{n}{append}\PY{p}{(}\PY{n}{sent\PYZus{}1}\PY{p}{)}
            \PY{n}{new\PYZus{}mix}\PY{o}{.}\PY{n}{append}\PY{p}{(}\PY{n}{sent\PYZus{}2}\PY{p}{)}
        \PY{k}{else}\PY{p}{:}
            \PY{n}{new\PYZus{}mix}\PY{o}{.}\PY{n}{append}\PY{p}{(}\PY{n}{sent}\PY{p}{)}

    \PY{n}{mix\PYZus{}sent} \PY{o}{=} \PY{n}{new\PYZus{}mix}

    \PY{n}{port} \PY{o}{=} \PY{p}{[}\PY{p}{]}
    \PY{n}{engl} \PY{o}{=} \PY{p}{[}\PY{p}{]}

    \PY{k}{for} \PY{n}{sent} \PY{o+ow}{in} \PY{n}{mix\PYZus{}sent}\PY{p}{:}
        \PY{k}{try}\PY{p}{:}
            \PY{k}{if} \PY{n}{detect} \PY{p}{(}\PY{n}{sent}\PY{p}{)} \PY{o}{==} \PY{l+s+s1}{\PYZsq{}}\PY{l+s+s1}{pt}\PY{l+s+s1}{\PYZsq{}}\PY{p}{:}
                \PY{n}{port}\PY{o}{.}\PY{n}{append}\PY{p}{(}\PY{n}{sent}\PY{p}{)}
            \PY{k}{else}\PY{p}{:}
                \PY{n}{engl}\PY{o}{.}\PY{n}{append} \PY{p}{(}\PY{n}{sent}\PY{p}{)}
        \PY{k}{except}\PY{p}{:}
            \PY{k}{pass}

    \PY{n}{port} \PY{o}{=} \PY{l+s+s2}{\PYZdq{}}\PY{l+s+s2}{ }\PY{l+s+s2}{\PYZdq{}}\PY{o}{.}\PY{n}{join}\PY{p}{(}\PY{n}{port}\PY{p}{)}
    \PY{n}{engl} \PY{o}{=} \PY{l+s+s2}{\PYZdq{}}\PY{l+s+s2}{ }\PY{l+s+s2}{\PYZdq{}}\PY{o}{.}\PY{n}{join}\PY{p}{(}\PY{n}{engl}\PY{p}{)}

    \PY{k}{return}\PY{p}{(}\PY{n}{port}\PY{p}{,} \PY{n}{engl}\PY{p}{)}
\end{Verbatim}
\end{tcolorbox}

    Até esse momento estamos recuperando as informações de todas as teses de
uma determinada instituição. No entanto o objetivo é gravar os metadados
e salvar o arquivo apenas das teses relacionadas a O\&G. Portanto, vamos
carregar os algoritmos de classificação e de vetorização de palavras
treinados previamente.

    \begin{tcolorbox}[breakable, size=fbox, boxrule=1pt, pad at break*=1mm,colback=cellbackground, colframe=cellborder]
\prompt{In}{incolor}{8}{\hspace{4pt}}
\begin{Verbatim}[commandchars=\\\{\}]
\PY{c+c1}{\PYZsh{} Carregando modelo Word2Vec}
\PY{n}{BDTD\PYZus{}word2vec\PYZus{}50} \PY{o}{=} \PY{n}{Word2Vec}\PY{o}{.}\PY{n}{load}\PY{p}{(}\PY{l+s+s2}{\PYZdq{}}\PY{l+s+s2}{..}\PY{l+s+s2}{\PYZbs{}}\PY{l+s+s2}{..}\PY{l+s+s2}{\PYZbs{}}\PY{l+s+s2}{..}\PY{l+s+s2}{\PYZbs{}}\PY{l+s+s2}{Embeddings}\PY{l+s+s2}{\PYZbs{}}\PY{l+s+s2}{BDTD\PYZus{}word2vec\PYZus{}50}\PY{l+s+s2}{\PYZdq{}}\PY{p}{)}
\PY{c+c1}{\PYZsh{} Carregando modelo keras}
\PY{n}{model\PYZus{}keras} \PY{o}{=} \PY{n}{load\PYZus{}model}\PY{p}{(}\PY{l+s+s1}{\PYZsq{}}\PY{l+s+s1}{..}\PY{l+s+s1}{\PYZbs{}}\PY{l+s+s1}{..}\PY{l+s+s1}{\PYZbs{}}\PY{l+s+s1}{..}\PY{l+s+s1}{\PYZbs{}}\PY{l+s+s1}{model\PYZus{}cnn.h5}\PY{l+s+s1}{\PYZsq{}}\PY{p}{)}
\PY{n}{model\PYZus{}keras}\PY{o}{.}\PY{n}{summary}\PY{p}{(}\PY{p}{)}
\end{Verbatim}
\end{tcolorbox}

    \begin{Verbatim}[commandchars=\\\{\}]
C:\textbackslash{}ProgramData\textbackslash{}Anaconda3\textbackslash{}lib\textbackslash{}site-packages\textbackslash{}smart\_open\textbackslash{}smart\_open\_lib.py:398:
UserWarning: This function is deprecated, use smart\_open.open instead. See the
migration notes for details: https://github.com/RaRe-
Technologies/smart\_open/blob/master/README.rst\#migrating-to-the-new-open-
function
  'See the migration notes for details: \%s' \% \_MIGRATION\_NOTES\_URL
\end{Verbatim}

    \begin{Verbatim}[commandchars=\\\{\}]
WARNING:tensorflow:From C:\textbackslash{}ProgramData\textbackslash{}Anaconda3\textbackslash{}lib\textbackslash{}site-
packages\textbackslash{}keras\textbackslash{}backend\textbackslash{}tensorflow\_backend.py:517: The name tf.placeholder is
deprecated. Please use tf.compat.v1.placeholder instead.

WARNING:tensorflow:From C:\textbackslash{}ProgramData\textbackslash{}Anaconda3\textbackslash{}lib\textbackslash{}site-
packages\textbackslash{}keras\textbackslash{}backend\textbackslash{}tensorflow\_backend.py:4138: The name tf.random\_uniform is
deprecated. Please use tf.random.uniform instead.

WARNING:tensorflow:From C:\textbackslash{}ProgramData\textbackslash{}Anaconda3\textbackslash{}lib\textbackslash{}site-
packages\textbackslash{}keras\textbackslash{}backend\textbackslash{}tensorflow\_backend.py:131: The name tf.get\_default\_graph
is deprecated. Please use tf.compat.v1.get\_default\_graph instead.

WARNING:tensorflow:From C:\textbackslash{}ProgramData\textbackslash{}Anaconda3\textbackslash{}lib\textbackslash{}site-
packages\textbackslash{}keras\textbackslash{}backend\textbackslash{}tensorflow\_backend.py:133: The name
tf.placeholder\_with\_default is deprecated. Please use
tf.compat.v1.placeholder\_with\_default instead.

WARNING:tensorflow:From C:\textbackslash{}ProgramData\textbackslash{}Anaconda3\textbackslash{}lib\textbackslash{}site-
packages\textbackslash{}keras\textbackslash{}backend\textbackslash{}tensorflow\_backend.py:3445: calling dropout (from
tensorflow.python.ops.nn\_ops) with keep\_prob is deprecated and will be removed
in a future version.
Instructions for updating:
Please use `rate` instead of `keep\_prob`. Rate should be set to `rate = 1 -
keep\_prob`.
WARNING:tensorflow:From C:\textbackslash{}ProgramData\textbackslash{}Anaconda3\textbackslash{}lib\textbackslash{}site-
packages\textbackslash{}keras\textbackslash{}backend\textbackslash{}tensorflow\_backend.py:3976: The name tf.nn.max\_pool is
deprecated. Please use tf.nn.max\_pool2d instead.

WARNING:tensorflow:From C:\textbackslash{}ProgramData\textbackslash{}Anaconda3\textbackslash{}lib\textbackslash{}site-
packages\textbackslash{}keras\textbackslash{}backend\textbackslash{}tensorflow\_backend.py:174: The name
tf.get\_default\_session is deprecated. Please use
tf.compat.v1.get\_default\_session instead.

WARNING:tensorflow:From C:\textbackslash{}ProgramData\textbackslash{}Anaconda3\textbackslash{}lib\textbackslash{}site-
packages\textbackslash{}keras\textbackslash{}optimizers.py:790: The name tf.train.Optimizer is deprecated.
Please use tf.compat.v1.train.Optimizer instead.

WARNING:tensorflow:From C:\textbackslash{}ProgramData\textbackslash{}Anaconda3\textbackslash{}lib\textbackslash{}site-
packages\textbackslash{}tensorflow\textbackslash{}python\textbackslash{}ops\textbackslash{}nn\_impl.py:180:
add\_dispatch\_support.<locals>.wrapper (from tensorflow.python.ops.array\_ops) is
deprecated and will be removed in a future version.
Instructions for updating:
Use tf.where in 2.0, which has the same broadcast rule as np.where
\_\_\_\_\_\_\_\_\_\_\_\_\_\_\_\_\_\_\_\_\_\_\_\_\_\_\_\_\_\_\_\_\_\_\_\_\_\_\_\_\_\_\_\_\_\_\_\_\_\_\_\_\_\_\_\_\_\_\_\_\_\_\_\_\_
Layer (type)                 Output Shape              Param \#
=================================================================
input\_6 (InputLayer)         (None, 400)               0
\_\_\_\_\_\_\_\_\_\_\_\_\_\_\_\_\_\_\_\_\_\_\_\_\_\_\_\_\_\_\_\_\_\_\_\_\_\_\_\_\_\_\_\_\_\_\_\_\_\_\_\_\_\_\_\_\_\_\_\_\_\_\_\_\_
embedding\_6 (Embedding)      (None, 400, 50)           9289150
\_\_\_\_\_\_\_\_\_\_\_\_\_\_\_\_\_\_\_\_\_\_\_\_\_\_\_\_\_\_\_\_\_\_\_\_\_\_\_\_\_\_\_\_\_\_\_\_\_\_\_\_\_\_\_\_\_\_\_\_\_\_\_\_\_
spatial\_dropout1d\_6 (Spatial (None, 400, 50)           0
\_\_\_\_\_\_\_\_\_\_\_\_\_\_\_\_\_\_\_\_\_\_\_\_\_\_\_\_\_\_\_\_\_\_\_\_\_\_\_\_\_\_\_\_\_\_\_\_\_\_\_\_\_\_\_\_\_\_\_\_\_\_\_\_\_
conv1d\_13 (Conv1D)           (None, 396, 128)          32128
\_\_\_\_\_\_\_\_\_\_\_\_\_\_\_\_\_\_\_\_\_\_\_\_\_\_\_\_\_\_\_\_\_\_\_\_\_\_\_\_\_\_\_\_\_\_\_\_\_\_\_\_\_\_\_\_\_\_\_\_\_\_\_\_\_
max\_pooling1d\_9 (MaxPooling1 (None, 198, 128)          0
\_\_\_\_\_\_\_\_\_\_\_\_\_\_\_\_\_\_\_\_\_\_\_\_\_\_\_\_\_\_\_\_\_\_\_\_\_\_\_\_\_\_\_\_\_\_\_\_\_\_\_\_\_\_\_\_\_\_\_\_\_\_\_\_\_
dropout\_18 (Dropout)         (None, 198, 128)          0
\_\_\_\_\_\_\_\_\_\_\_\_\_\_\_\_\_\_\_\_\_\_\_\_\_\_\_\_\_\_\_\_\_\_\_\_\_\_\_\_\_\_\_\_\_\_\_\_\_\_\_\_\_\_\_\_\_\_\_\_\_\_\_\_\_
conv1d\_14 (Conv1D)           (None, 194, 128)          82048
\_\_\_\_\_\_\_\_\_\_\_\_\_\_\_\_\_\_\_\_\_\_\_\_\_\_\_\_\_\_\_\_\_\_\_\_\_\_\_\_\_\_\_\_\_\_\_\_\_\_\_\_\_\_\_\_\_\_\_\_\_\_\_\_\_
max\_pooling1d\_10 (MaxPooling (None, 97, 128)           0
\_\_\_\_\_\_\_\_\_\_\_\_\_\_\_\_\_\_\_\_\_\_\_\_\_\_\_\_\_\_\_\_\_\_\_\_\_\_\_\_\_\_\_\_\_\_\_\_\_\_\_\_\_\_\_\_\_\_\_\_\_\_\_\_\_
dropout\_19 (Dropout)         (None, 97, 128)           0
\_\_\_\_\_\_\_\_\_\_\_\_\_\_\_\_\_\_\_\_\_\_\_\_\_\_\_\_\_\_\_\_\_\_\_\_\_\_\_\_\_\_\_\_\_\_\_\_\_\_\_\_\_\_\_\_\_\_\_\_\_\_\_\_\_
conv1d\_15 (Conv1D)           (None, 93, 128)           82048
\_\_\_\_\_\_\_\_\_\_\_\_\_\_\_\_\_\_\_\_\_\_\_\_\_\_\_\_\_\_\_\_\_\_\_\_\_\_\_\_\_\_\_\_\_\_\_\_\_\_\_\_\_\_\_\_\_\_\_\_\_\_\_\_\_
global\_max\_pooling1d\_5 (Glob (None, 128)               0
\_\_\_\_\_\_\_\_\_\_\_\_\_\_\_\_\_\_\_\_\_\_\_\_\_\_\_\_\_\_\_\_\_\_\_\_\_\_\_\_\_\_\_\_\_\_\_\_\_\_\_\_\_\_\_\_\_\_\_\_\_\_\_\_\_
dropout\_20 (Dropout)         (None, 128)               0
\_\_\_\_\_\_\_\_\_\_\_\_\_\_\_\_\_\_\_\_\_\_\_\_\_\_\_\_\_\_\_\_\_\_\_\_\_\_\_\_\_\_\_\_\_\_\_\_\_\_\_\_\_\_\_\_\_\_\_\_\_\_\_\_\_
dense\_30 (Dense)             (None, 512)               66048
\_\_\_\_\_\_\_\_\_\_\_\_\_\_\_\_\_\_\_\_\_\_\_\_\_\_\_\_\_\_\_\_\_\_\_\_\_\_\_\_\_\_\_\_\_\_\_\_\_\_\_\_\_\_\_\_\_\_\_\_\_\_\_\_\_
dense\_31 (Dense)             (None, 512)               262656
\_\_\_\_\_\_\_\_\_\_\_\_\_\_\_\_\_\_\_\_\_\_\_\_\_\_\_\_\_\_\_\_\_\_\_\_\_\_\_\_\_\_\_\_\_\_\_\_\_\_\_\_\_\_\_\_\_\_\_\_\_\_\_\_\_
dropout\_21 (Dropout)         (None, 512)               0
\_\_\_\_\_\_\_\_\_\_\_\_\_\_\_\_\_\_\_\_\_\_\_\_\_\_\_\_\_\_\_\_\_\_\_\_\_\_\_\_\_\_\_\_\_\_\_\_\_\_\_\_\_\_\_\_\_\_\_\_\_\_\_\_\_
dense\_32 (Dense)             (None, 1)                 513
=================================================================
Total params: 9,814,591
Trainable params: 525,441
Non-trainable params: 9,289,150
\_\_\_\_\_\_\_\_\_\_\_\_\_\_\_\_\_\_\_\_\_\_\_\_\_\_\_\_\_\_\_\_\_\_\_\_\_\_\_\_\_\_\_\_\_\_\_\_\_\_\_\_\_\_\_\_\_\_\_\_\_\_\_\_\_
\end{Verbatim}

    \begin{tcolorbox}[breakable, size=fbox, boxrule=1pt, pad at break*=1mm,colback=cellbackground, colframe=cellborder]
\prompt{In}{incolor}{9}{\hspace{4pt}}
\begin{Verbatim}[commandchars=\\\{\}]
\PY{c+c1}{\PYZsh{} dicionário proveniente do modelo de word embedding para converter palavras em índices}
\PY{n}{word2index} \PY{o}{=} \PY{p}{\PYZob{}}\PY{p}{\PYZcb{}}
\PY{k}{for} \PY{n}{index}\PY{p}{,} \PY{n}{word} \PY{o+ow}{in} \PY{n+nb}{enumerate}\PY{p}{(}\PY{n}{BDTD\PYZus{}word2vec\PYZus{}50}\PY{o}{.}\PY{n}{wv}\PY{o}{.}\PY{n}{index2word}\PY{p}{)}\PY{p}{:}
    \PY{n}{word2index}\PY{p}{[}\PY{n}{word}\PY{p}{]} \PY{o}{=} \PY{n}{index}
    
\PY{c+c1}{\PYZsh{} Função para converter texto em sequência de índices}
\PY{k}{def} \PY{n+nf}{index\PYZus{}pad\PYZus{}text}\PY{p}{(}\PY{n}{text}\PY{p}{,} \PY{n}{maxlen}\PY{p}{,} \PY{n}{word2index}\PY{p}{)}\PY{p}{:}
    \PY{n}{maxlen} \PY{o}{=} \PY{l+m+mi}{400}
    \PY{n}{new\PYZus{}text} \PY{o}{=} \PY{p}{[}\PY{p}{]} 
    
    \PY{k}{for} \PY{n}{word} \PY{o+ow}{in} \PY{n}{word\PYZus{}tokenize}\PY{p}{(}\PY{n}{text}\PY{p}{)}\PY{p}{:}
        \PY{k}{try}\PY{p}{:}
            \PY{n}{new\PYZus{}text}\PY{o}{.}\PY{n}{append}\PY{p}{(}\PY{n}{word2index}\PY{p}{[}\PY{n}{word}\PY{p}{]}\PY{p}{)}
        \PY{k}{except}\PY{p}{:}
            \PY{k}{pass}
    \PY{c+c1}{\PYZsh{} Add the padding for each sentence. Here I am padding with 0}
    \PY{k}{if} \PY{n+nb}{len}\PY{p}{(}\PY{n}{new\PYZus{}text}\PY{p}{)} \PY{o}{\PYZgt{}} \PY{n}{maxlen}\PY{p}{:}
        \PY{n}{new\PYZus{}text} \PY{o}{=} \PY{n}{new\PYZus{}text}\PY{p}{[}\PY{p}{:}\PY{l+m+mi}{400}\PY{p}{]}
    \PY{k}{else}\PY{p}{:}
        \PY{n}{new\PYZus{}text} \PY{o}{+}\PY{o}{=} \PY{p}{[}\PY{l+m+mi}{0}\PY{p}{]} \PY{o}{*} \PY{p}{(}\PY{n}{maxlen} \PY{o}{\PYZhy{}} \PY{n+nb}{len}\PY{p}{(}\PY{n}{new\PYZus{}text}\PY{p}{)}\PY{p}{)}

    \PY{k}{return} \PY{n}{np}\PY{o}{.}\PY{n}{array}\PY{p}{(}\PY{n}{new\PYZus{}text}\PY{p}{)}

\PY{n}{maxlen} \PY{o}{=} \PY{l+m+mi}{400}
\end{Verbatim}
\end{tcolorbox}

    Para cada link coletado será fita as seguintes tarefas: * verificar se o
texto português e inglês estão misturados; * transformar o texto em
sequência de índices; * classificar quanto a relevância ao domínio de
O\&G; * se for relevante, gravar os metadados

    \begin{tcolorbox}[breakable, size=fbox, boxrule=1pt, pad at break*=1mm,colback=cellbackground, colframe=cellborder]
\prompt{In}{incolor}{10}{\hspace{4pt}}
\begin{Verbatim}[commandchars=\\\{\}]
\PY{c+c1}{\PYZsh{} Dicionário para agrupar os metadados}
\PY{n}{metadados} \PY{o}{=} \PY{p}{\PYZob{}}\PY{p}{\PYZcb{}}
\PY{c+c1}{\PYZsh{} Contadores te links testados e classificados como O\PYZam{}G}
\PY{n}{n\PYZus{}test} \PY{o}{=} \PY{l+m+mi}{0}
\PY{n}{n\PYZus{}pet} \PY{o}{=} \PY{l+m+mi}{0}
\PY{c+c1}{\PYZsh{} Testando cada link de links}
\PY{k}{for} \PY{n}{link} \PY{o+ow}{in} \PY{n}{links}\PY{p}{:}
    \PY{n}{n\PYZus{}test} \PY{o}{+}\PY{o}{=} \PY{l+m+mi}{1}
    \PY{k}{try}\PY{p}{:}
        \PY{c+c1}{\PYZsh{} Recuperar o metadados de uma tese}
        \PY{n}{metadado} \PY{o}{=} \PY{n}{tese\PYZus{}link}\PY{p}{(}\PY{n}{link}\PY{p}{)}
        \PY{c+c1}{\PYZsh{} Verificar se existe resumo em inglês, separar texto português/inglês e realocar }
        \PY{c+c1}{\PYZsh{} os textos separados nas respectivas colunas}
        \PY{k}{if} \PY{l+s+s1}{\PYZsq{}}\PY{l+s+s1}{Resumo inglês:}\PY{l+s+s1}{\PYZsq{}} \PY{o+ow}{not} \PY{o+ow}{in} \PY{n}{metadado}\PY{p}{:}
            \PY{n}{metadado}\PY{p}{[}\PY{l+s+s1}{\PYZsq{}}\PY{l+s+s1}{Resumo inglês:}\PY{l+s+s1}{\PYZsq{}}\PY{p}{]} \PY{o}{=} \PY{n}{separacao\PYZus{}port\PYZus{}engl}\PY{p}{(}\PY{n}{metadado}\PY{p}{[}\PY{l+s+s1}{\PYZsq{}}\PY{l+s+s1}{Resumo Português:}\PY{l+s+s1}{\PYZsq{}}\PY{p}{]}\PY{p}{)}\PY{p}{[}\PY{l+m+mi}{1}\PY{p}{]}
        \PY{n}{metadado}\PY{p}{[}\PY{l+s+s1}{\PYZsq{}}\PY{l+s+s1}{Resumo Português:}\PY{l+s+s1}{\PYZsq{}}\PY{p}{]} \PY{o}{=} \PY{n}{separacao\PYZus{}port\PYZus{}engl}\PY{p}{(}\PY{n}{metadado}\PY{p}{[}\PY{l+s+s1}{\PYZsq{}}\PY{l+s+s1}{Resumo Português:}\PY{l+s+s1}{\PYZsq{}}\PY{p}{]}\PY{p}{)}\PY{p}{[}\PY{l+m+mi}{0}\PY{p}{]}
        \PY{c+c1}{\PYZsh{} Colocando o texto em minúscula}
        \PY{n}{text} \PY{o}{=} \PY{n}{metadado}\PY{p}{[}\PY{l+s+s1}{\PYZsq{}}\PY{l+s+s1}{Resumo Português:}\PY{l+s+s1}{\PYZsq{}}\PY{p}{]}\PY{o}{.}\PY{n}{lower}\PY{p}{(}\PY{p}{)}
        \PY{c+c1}{\PYZsh{} Convertendo as palavras em sequencias de acordo com o modelo word2vec}
        \PY{n}{text\PYZus{}seq} \PY{o}{=} \PY{n}{index\PYZus{}pad\PYZus{}text}\PY{p}{(}\PY{n}{text}\PY{p}{,} \PY{n}{maxlen}\PY{p}{,} \PY{n}{word2index}\PY{p}{)}
        \PY{n}{text\PYZus{}seq} \PY{o}{=} \PY{n}{text\PYZus{}seq}\PY{o}{.}\PY{n}{reshape}\PY{p}{(}\PY{p}{(}\PY{l+m+mi}{1}\PY{p}{,} \PY{l+m+mi}{400}\PY{p}{)}\PY{p}{)}
        \PY{c+c1}{\PYZsh{} Usando o algoritmo classificador para prever se a tese é relevante}
        \PY{n}{pred} \PY{o}{=} \PY{n}{model\PYZus{}keras}\PY{o}{.}\PY{n}{predict}\PY{p}{(}\PY{n}{text\PYZus{}seq}\PY{p}{)}\PY{p}{[}\PY{l+m+mi}{0}\PY{p}{]}
        \PY{c+c1}{\PYZsh{}Se a classificação for menor do que 0.2 manter os metadados}
        \PY{k}{if} \PY{p}{(}\PY{n}{pred} \PY{o}{\PYZlt{}} \PY{l+m+mf}{0.2} \PY{o+ow}{and} \PY{n+nb}{len}\PY{p}{(}\PY{n}{text}\PY{p}{)} \PY{o}{\PYZgt{}} \PY{l+m+mi}{100}\PY{p}{)}\PY{p}{:}
            \PY{n}{metadado}\PY{p}{[}\PY{l+s+s1}{\PYZsq{}}\PY{l+s+s1}{Classificador}\PY{l+s+s1}{\PYZsq{}}\PY{p}{]} \PY{o}{=} \PY{n}{pred}\PY{p}{[}\PY{l+m+mi}{0}\PY{p}{]}
            \PY{n}{texto\PYZus{}completo} \PY{o}{=} \PY{n}{metadado}\PY{p}{[}\PY{l+s+s1}{\PYZsq{}}\PY{l+s+s1}{Download Texto Completo:}\PY{l+s+s1}{\PYZsq{}}\PY{p}{]}
            \PY{n}{metadados}\PY{p}{[}\PY{n}{texto\PYZus{}completo}\PY{p}{]} \PY{o}{=} \PY{n}{metadado}
            \PY{n}{n\PYZus{}pet} \PY{o}{+}\PY{o}{=} \PY{l+m+mi}{1}
            \PY{c+c1}{\PYZsh{} Gravando os resultados em JSON}
            \PY{n}{metadados\PYZus{}ufba} \PY{o}{=} \PY{n}{pd}\PY{o}{.}\PY{n}{DataFrame}\PY{o}{.}\PY{n}{from\PYZus{}dict}\PY{p}{(}\PY{n}{metadados}\PY{p}{,} \PY{n}{orient}\PY{o}{=}\PY{l+s+s1}{\PYZsq{}}\PY{l+s+s1}{index}\PY{l+s+s1}{\PYZsq{}}\PY{p}{)}
            \PY{n}{metadados\PYZus{}ufba}\PY{o}{.}\PY{n}{to\PYZus{}json}\PY{p}{(}\PY{l+s+s1}{\PYZsq{}}\PY{l+s+s1}{metadados\PYZus{}ufba.json}\PY{l+s+s1}{\PYZsq{}}\PY{p}{,} \PY{n}{orient} \PY{o}{=} \PY{l+s+s1}{\PYZsq{}}\PY{l+s+s1}{index}\PY{l+s+s1}{\PYZsq{}}\PY{p}{)}
            \PY{n+nb}{print}\PY{p}{(}\PY{n}{n\PYZus{}test}\PY{p}{,} \PY{l+s+s2}{\PYZdq{}}\PY{l+s+s2}{ teses avaliadas e }\PY{l+s+s2}{\PYZdq{}}\PY{p}{,} \PY{n}{n\PYZus{}pet}\PY{p}{,} \PY{l+s+s2}{\PYZdq{}}\PY{l+s+s2}{ teses relacionadas a O\PYZam{}G encontradas.}\PY{l+s+s2}{\PYZdq{}}\PY{p}{)}
    
    \PY{k}{except}\PY{p}{:}
        \PY{k}{pass}
    
    
\end{Verbatim}
\end{tcolorbox}

    \begin{Verbatim}[commandchars=\\\{\}]
27  teses avaliadas e  1  teses relacionadas a O\&G encontradas.
37  teses avaliadas e  2  teses relacionadas a O\&G encontradas.
132  teses avaliadas e  3  teses relacionadas a O\&G encontradas.
173  teses avaliadas e  4  teses relacionadas a O\&G encontradas.
218  teses avaliadas e  5  teses relacionadas a O\&G encontradas.
241  teses avaliadas e  6  teses relacionadas a O\&G encontradas.
343  teses avaliadas e  7  teses relacionadas a O\&G encontradas.
427  teses avaliadas e  8  teses relacionadas a O\&G encontradas.
448  teses avaliadas e  9  teses relacionadas a O\&G encontradas.
467  teses avaliadas e  10  teses relacionadas a O\&G encontradas.
488  teses avaliadas e  11  teses relacionadas a O\&G encontradas.
597  teses avaliadas e  12  teses relacionadas a O\&G encontradas.
727  teses avaliadas e  13  teses relacionadas a O\&G encontradas.
823  teses avaliadas e  14  teses relacionadas a O\&G encontradas.
862  teses avaliadas e  15  teses relacionadas a O\&G encontradas.
865  teses avaliadas e  16  teses relacionadas a O\&G encontradas.
868  teses avaliadas e  17  teses relacionadas a O\&G encontradas.
1012  teses avaliadas e  18  teses relacionadas a O\&G encontradas.
1017  teses avaliadas e  19  teses relacionadas a O\&G encontradas.
1022  teses avaliadas e  20  teses relacionadas a O\&G encontradas.
1025  teses avaliadas e  21  teses relacionadas a O\&G encontradas.
1068  teses avaliadas e  22  teses relacionadas a O\&G encontradas.
1093  teses avaliadas e  23  teses relacionadas a O\&G encontradas.
1094  teses avaliadas e  24  teses relacionadas a O\&G encontradas.
1178  teses avaliadas e  25  teses relacionadas a O\&G encontradas.
1190  teses avaliadas e  26  teses relacionadas a O\&G encontradas.
1192  teses avaliadas e  27  teses relacionadas a O\&G encontradas.
1195  teses avaliadas e  28  teses relacionadas a O\&G encontradas.
1196  teses avaliadas e  29  teses relacionadas a O\&G encontradas.
1368  teses avaliadas e  30  teses relacionadas a O\&G encontradas.
1374  teses avaliadas e  31  teses relacionadas a O\&G encontradas.
1385  teses avaliadas e  32  teses relacionadas a O\&G encontradas.
1413  teses avaliadas e  33  teses relacionadas a O\&G encontradas.
1431  teses avaliadas e  34  teses relacionadas a O\&G encontradas.
1435  teses avaliadas e  35  teses relacionadas a O\&G encontradas.
1475  teses avaliadas e  36  teses relacionadas a O\&G encontradas.
1508  teses avaliadas e  37  teses relacionadas a O\&G encontradas.
1761  teses avaliadas e  38  teses relacionadas a O\&G encontradas.
1880  teses avaliadas e  39  teses relacionadas a O\&G encontradas.
1921  teses avaliadas e  40  teses relacionadas a O\&G encontradas.
1972  teses avaliadas e  41  teses relacionadas a O\&G encontradas.
2010  teses avaliadas e  42  teses relacionadas a O\&G encontradas.
2163  teses avaliadas e  43  teses relacionadas a O\&G encontradas.
2196  teses avaliadas e  44  teses relacionadas a O\&G encontradas.
2231  teses avaliadas e  45  teses relacionadas a O\&G encontradas.
2232  teses avaliadas e  46  teses relacionadas a O\&G encontradas.
2234  teses avaliadas e  47  teses relacionadas a O\&G encontradas.
2388  teses avaliadas e  48  teses relacionadas a O\&G encontradas.
2413  teses avaliadas e  49  teses relacionadas a O\&G encontradas.
2416  teses avaliadas e  50  teses relacionadas a O\&G encontradas.
2418  teses avaliadas e  51  teses relacionadas a O\&G encontradas.
2420  teses avaliadas e  52  teses relacionadas a O\&G encontradas.
2424  teses avaliadas e  53  teses relacionadas a O\&G encontradas.
2427  teses avaliadas e  54  teses relacionadas a O\&G encontradas.
3101  teses avaliadas e  55  teses relacionadas a O\&G encontradas.
3293  teses avaliadas e  56  teses relacionadas a O\&G encontradas.
3295  teses avaliadas e  57  teses relacionadas a O\&G encontradas.
3297  teses avaliadas e  58  teses relacionadas a O\&G encontradas.
3385  teses avaliadas e  59  teses relacionadas a O\&G encontradas.
3396  teses avaliadas e  60  teses relacionadas a O\&G encontradas.
3404  teses avaliadas e  61  teses relacionadas a O\&G encontradas.
3405  teses avaliadas e  62  teses relacionadas a O\&G encontradas.
3415  teses avaliadas e  63  teses relacionadas a O\&G encontradas.
3752  teses avaliadas e  64  teses relacionadas a O\&G encontradas.
3794  teses avaliadas e  65  teses relacionadas a O\&G encontradas.
3844  teses avaliadas e  66  teses relacionadas a O\&G encontradas.
3871  teses avaliadas e  67  teses relacionadas a O\&G encontradas.
3880  teses avaliadas e  68  teses relacionadas a O\&G encontradas.
3902  teses avaliadas e  69  teses relacionadas a O\&G encontradas.
3917  teses avaliadas e  70  teses relacionadas a O\&G encontradas.
3920  teses avaliadas e  71  teses relacionadas a O\&G encontradas.
4003  teses avaliadas e  72  teses relacionadas a O\&G encontradas.
4013  teses avaliadas e  73  teses relacionadas a O\&G encontradas.
4040  teses avaliadas e  74  teses relacionadas a O\&G encontradas.
4082  teses avaliadas e  75  teses relacionadas a O\&G encontradas.
4129  teses avaliadas e  76  teses relacionadas a O\&G encontradas.
4130  teses avaliadas e  77  teses relacionadas a O\&G encontradas.
4181  teses avaliadas e  78  teses relacionadas a O\&G encontradas.
4220  teses avaliadas e  79  teses relacionadas a O\&G encontradas.
4229  teses avaliadas e  80  teses relacionadas a O\&G encontradas.
4232  teses avaliadas e  81  teses relacionadas a O\&G encontradas.
4260  teses avaliadas e  82  teses relacionadas a O\&G encontradas.
4300  teses avaliadas e  83  teses relacionadas a O\&G encontradas.
4447  teses avaliadas e  84  teses relacionadas a O\&G encontradas.
4449  teses avaliadas e  85  teses relacionadas a O\&G encontradas.
4515  teses avaliadas e  86  teses relacionadas a O\&G encontradas.
4517  teses avaliadas e  87  teses relacionadas a O\&G encontradas.
4519  teses avaliadas e  88  teses relacionadas a O\&G encontradas.
4566  teses avaliadas e  89  teses relacionadas a O\&G encontradas.
4569  teses avaliadas e  90  teses relacionadas a O\&G encontradas.
4579  teses avaliadas e  91  teses relacionadas a O\&G encontradas.
4582  teses avaliadas e  92  teses relacionadas a O\&G encontradas.
4655  teses avaliadas e  93  teses relacionadas a O\&G encontradas.
4842  teses avaliadas e  94  teses relacionadas a O\&G encontradas.
4850  teses avaliadas e  95  teses relacionadas a O\&G encontradas.
4855  teses avaliadas e  96  teses relacionadas a O\&G encontradas.
4891  teses avaliadas e  97  teses relacionadas a O\&G encontradas.
4900  teses avaliadas e  98  teses relacionadas a O\&G encontradas.
4957  teses avaliadas e  99  teses relacionadas a O\&G encontradas.
5039  teses avaliadas e  100  teses relacionadas a O\&G encontradas.
5091  teses avaliadas e  101  teses relacionadas a O\&G encontradas.
5173  teses avaliadas e  102  teses relacionadas a O\&G encontradas.
5174  teses avaliadas e  103  teses relacionadas a O\&G encontradas.
5183  teses avaliadas e  104  teses relacionadas a O\&G encontradas.
5199  teses avaliadas e  105  teses relacionadas a O\&G encontradas.
5223  teses avaliadas e  106  teses relacionadas a O\&G encontradas.
5361  teses avaliadas e  107  teses relacionadas a O\&G encontradas.
5400  teses avaliadas e  108  teses relacionadas a O\&G encontradas.
5499  teses avaliadas e  109  teses relacionadas a O\&G encontradas.
5511  teses avaliadas e  110  teses relacionadas a O\&G encontradas.
5539  teses avaliadas e  111  teses relacionadas a O\&G encontradas.
5540  teses avaliadas e  112  teses relacionadas a O\&G encontradas.
5626  teses avaliadas e  113  teses relacionadas a O\&G encontradas.
5638  teses avaliadas e  114  teses relacionadas a O\&G encontradas.
5661  teses avaliadas e  115  teses relacionadas a O\&G encontradas.
5664  teses avaliadas e  116  teses relacionadas a O\&G encontradas.
5667  teses avaliadas e  117  teses relacionadas a O\&G encontradas.
5724  teses avaliadas e  118  teses relacionadas a O\&G encontradas.
5775  teses avaliadas e  119  teses relacionadas a O\&G encontradas.
5847  teses avaliadas e  120  teses relacionadas a O\&G encontradas.
5850  teses avaliadas e  121  teses relacionadas a O\&G encontradas.
5885  teses avaliadas e  122  teses relacionadas a O\&G encontradas.
5910  teses avaliadas e  123  teses relacionadas a O\&G encontradas.
5929  teses avaliadas e  124  teses relacionadas a O\&G encontradas.
5930  teses avaliadas e  125  teses relacionadas a O\&G encontradas.
5931  teses avaliadas e  126  teses relacionadas a O\&G encontradas.
5971  teses avaliadas e  127  teses relacionadas a O\&G encontradas.
5974  teses avaliadas e  128  teses relacionadas a O\&G encontradas.
5990  teses avaliadas e  129  teses relacionadas a O\&G encontradas.
5991  teses avaliadas e  130  teses relacionadas a O\&G encontradas.
6178  teses avaliadas e  131  teses relacionadas a O\&G encontradas.
6210  teses avaliadas e  132  teses relacionadas a O\&G encontradas.
6242  teses avaliadas e  133  teses relacionadas a O\&G encontradas.
6315  teses avaliadas e  134  teses relacionadas a O\&G encontradas.
6322  teses avaliadas e  135  teses relacionadas a O\&G encontradas.
6324  teses avaliadas e  136  teses relacionadas a O\&G encontradas.
6408  teses avaliadas e  137  teses relacionadas a O\&G encontradas.
6409  teses avaliadas e  138  teses relacionadas a O\&G encontradas.
6531  teses avaliadas e  139  teses relacionadas a O\&G encontradas.
6806  teses avaliadas e  140  teses relacionadas a O\&G encontradas.
6920  teses avaliadas e  141  teses relacionadas a O\&G encontradas.
6951  teses avaliadas e  142  teses relacionadas a O\&G encontradas.
6953  teses avaliadas e  143  teses relacionadas a O\&G encontradas.
6954  teses avaliadas e  144  teses relacionadas a O\&G encontradas.
7063  teses avaliadas e  145  teses relacionadas a O\&G encontradas.
7147  teses avaliadas e  146  teses relacionadas a O\&G encontradas.
7161  teses avaliadas e  147  teses relacionadas a O\&G encontradas.
7327  teses avaliadas e  148  teses relacionadas a O\&G encontradas.
7498  teses avaliadas e  149  teses relacionadas a O\&G encontradas.
7665  teses avaliadas e  150  teses relacionadas a O\&G encontradas.
7736  teses avaliadas e  151  teses relacionadas a O\&G encontradas.
7795  teses avaliadas e  152  teses relacionadas a O\&G encontradas.
7796  teses avaliadas e  153  teses relacionadas a O\&G encontradas.
7800  teses avaliadas e  154  teses relacionadas a O\&G encontradas.
7805  teses avaliadas e  155  teses relacionadas a O\&G encontradas.
8027  teses avaliadas e  156  teses relacionadas a O\&G encontradas.
8082  teses avaliadas e  157  teses relacionadas a O\&G encontradas.
8086  teses avaliadas e  158  teses relacionadas a O\&G encontradas.
8103  teses avaliadas e  159  teses relacionadas a O\&G encontradas.
8248  teses avaliadas e  160  teses relacionadas a O\&G encontradas.
8442  teses avaliadas e  161  teses relacionadas a O\&G encontradas.
8653  teses avaliadas e  162  teses relacionadas a O\&G encontradas.
8686  teses avaliadas e  163  teses relacionadas a O\&G encontradas.
8688  teses avaliadas e  164  teses relacionadas a O\&G encontradas.
8705  teses avaliadas e  165  teses relacionadas a O\&G encontradas.
8724  teses avaliadas e  166  teses relacionadas a O\&G encontradas.
8728  teses avaliadas e  167  teses relacionadas a O\&G encontradas.
8783  teses avaliadas e  168  teses relacionadas a O\&G encontradas.
8794  teses avaliadas e  169  teses relacionadas a O\&G encontradas.
8848  teses avaliadas e  170  teses relacionadas a O\&G encontradas.
8849  teses avaliadas e  171  teses relacionadas a O\&G encontradas.
8850  teses avaliadas e  172  teses relacionadas a O\&G encontradas.
8854  teses avaliadas e  173  teses relacionadas a O\&G encontradas.
8857  teses avaliadas e  174  teses relacionadas a O\&G encontradas.
8858  teses avaliadas e  175  teses relacionadas a O\&G encontradas.
8955  teses avaliadas e  176  teses relacionadas a O\&G encontradas.
8965  teses avaliadas e  177  teses relacionadas a O\&G encontradas.
8981  teses avaliadas e  178  teses relacionadas a O\&G encontradas.
9036  teses avaliadas e  179  teses relacionadas a O\&G encontradas.
9230  teses avaliadas e  180  teses relacionadas a O\&G encontradas.
9235  teses avaliadas e  181  teses relacionadas a O\&G encontradas.
9239  teses avaliadas e  182  teses relacionadas a O\&G encontradas.
9251  teses avaliadas e  183  teses relacionadas a O\&G encontradas.
9260  teses avaliadas e  184  teses relacionadas a O\&G encontradas.
9379  teses avaliadas e  185  teses relacionadas a O\&G encontradas.
9380  teses avaliadas e  186  teses relacionadas a O\&G encontradas.
9531  teses avaliadas e  187  teses relacionadas a O\&G encontradas.
\end{Verbatim}

    \begin{tcolorbox}[breakable, size=fbox, boxrule=1pt, pad at break*=1mm,colback=cellbackground, colframe=cellborder]
\prompt{In}{incolor}{11}{\hspace{4pt}}
\begin{Verbatim}[commandchars=\\\{\}]
\PY{c+c1}{\PYZsh{} Incluindo um ID para cada tese}
\PY{n}{universidade} \PY{o}{=} \PY{l+s+s1}{\PYZsq{}}\PY{l+s+s1}{UFBA}\PY{l+s+s1}{\PYZsq{}}
\PY{n}{metadados\PYZus{}ufba}\PY{p}{[}\PY{l+s+s1}{\PYZsq{}}\PY{l+s+s1}{PDF\PYZus{}ID}\PY{l+s+s1}{\PYZsq{}}\PY{p}{]} \PY{o}{=} \PY{n}{metadados\PYZus{}ufba}\PY{p}{[}\PY{l+s+s1}{\PYZsq{}}\PY{l+s+s1}{Download Texto Completo:}\PY{l+s+s1}{\PYZsq{}}\PY{p}{]}\PY{o}{.}\PY{n}{apply}\PY{p}{(}\PY{k}{lambda} \PY{n}{x}\PY{p}{:} \PY{n}{universidade} \PY{o}{+} 
                                                                            \PY{l+s+s1}{\PYZsq{}}\PY{l+s+s1}{\PYZus{}}\PY{l+s+s1}{\PYZsq{}} \PY{o}{+} 
                                                                            \PY{n}{re}\PY{o}{.}\PY{n}{sub}\PY{p}{(}\PY{l+s+s1}{\PYZsq{}}\PY{l+s+s1}{/}\PY{l+s+s1}{\PYZsq{}}\PY{p}{,} \PY{l+s+s1}{\PYZsq{}}\PY{l+s+s1}{\PYZus{}}\PY{l+s+s1}{\PYZsq{}}\PY{p}{,} \PY{n}{x}\PY{p}{[}\PY{o}{\PYZhy{}}\PY{l+m+mi}{6}\PY{p}{:}\PY{p}{]}\PY{p}{)}\PY{p}{)}
\end{Verbatim}
\end{tcolorbox}

    \begin{tcolorbox}[breakable, size=fbox, boxrule=1pt, pad at break*=1mm,colback=cellbackground, colframe=cellborder]
\prompt{In}{incolor}{12}{\hspace{4pt}}
\begin{Verbatim}[commandchars=\\\{\}]
\PY{n}{metadados\PYZus{}ufba}\PY{o}{.}\PY{n}{to\PYZus{}json}\PY{p}{(}\PY{l+s+s1}{\PYZsq{}}\PY{l+s+s1}{metadados\PYZus{}ufba.json}\PY{l+s+s1}{\PYZsq{}}\PY{p}{,} \PY{n}{orient} \PY{o}{=} \PY{l+s+s1}{\PYZsq{}}\PY{l+s+s1}{index}\PY{l+s+s1}{\PYZsq{}}\PY{p}{)}
\end{Verbatim}
\end{tcolorbox}

    \begin{tcolorbox}[breakable, size=fbox, boxrule=1pt, pad at break*=1mm,colback=cellbackground, colframe=cellborder]
\prompt{In}{incolor}{13}{\hspace{4pt}}
\begin{Verbatim}[commandchars=\\\{\}]
\PY{c+c1}{\PYZsh{} Carregando arquivos já gravados}
\PY{n}{metadados\PYZus{}ufba} \PY{o}{=} \PY{n}{pd}\PY{o}{.}\PY{n}{read\PYZus{}json}\PY{p}{(}\PY{l+s+s1}{\PYZsq{}}\PY{l+s+s1}{metadados\PYZus{}ufba.json}\PY{l+s+s1}{\PYZsq{}}\PY{p}{,} \PY{n}{orient} \PY{o}{=} \PY{l+s+s1}{\PYZsq{}}\PY{l+s+s1}{index}\PY{l+s+s1}{\PYZsq{}}\PY{p}{)}
\end{Verbatim}
\end{tcolorbox}

    A próxima etapa será fazer o download das teses classificadas como
relevante para o domínio de O\&G

    \begin{tcolorbox}[breakable, size=fbox, boxrule=1pt, pad at break*=1mm,colback=cellbackground, colframe=cellborder]
\prompt{In}{incolor}{14}{\hspace{4pt}}
\begin{Verbatim}[commandchars=\\\{\}]
\PY{k}{for} \PY{n}{tese} \PY{o+ow}{in} \PY{n}{metadados\PYZus{}ufba}\PY{o}{.}\PY{n}{iterrows}\PY{p}{(}\PY{p}{)}\PY{p}{:}
    \PY{n+nb}{print}\PY{p}{(}\PY{n}{tese}\PY{p}{[}\PY{l+m+mi}{1}\PY{p}{]}\PY{p}{[}\PY{l+s+s1}{\PYZsq{}}\PY{l+s+s1}{PDF\PYZus{}ID}\PY{l+s+s1}{\PYZsq{}}\PY{p}{]}\PY{p}{)}
    \PY{k}{try}\PY{p}{:}
        \PY{c+c1}{\PYZsh{}preparar a url}
        \PY{n}{url} \PY{o}{=} \PY{n}{tese}\PY{p}{[}\PY{l+m+mi}{1}\PY{p}{]}\PY{p}{[}\PY{l+s+s1}{\PYZsq{}}\PY{l+s+s1}{Download Texto Completo:}\PY{l+s+s1}{\PYZsq{}}\PY{p}{]}

        \PY{c+c1}{\PYZsh{}Fazer requisição e parsear o arquivo html}
        \PY{n}{f} \PY{o}{=} \PY{n}{requests}\PY{o}{.}\PY{n}{get}\PY{p}{(}\PY{n}{url}\PY{p}{,} \PY{n}{proxies} \PY{o}{=} \PY{n}{proxies}\PY{p}{)}\PY{o}{.}\PY{n}{text} 
        \PY{n}{soup} \PY{o}{=} \PY{n}{bs}\PY{p}{(}\PY{n}{f}\PY{p}{,} \PY{l+s+s2}{\PYZdq{}}\PY{l+s+s2}{html.parser}\PY{l+s+s2}{\PYZdq{}}\PY{p}{)}

        \PY{c+c1}{\PYZsh{}Coletando link para arquivo das teses}
        \PY{n}{links} \PY{o}{=} \PY{p}{[}\PY{p}{]}
        \PY{k}{for} \PY{n}{doc} \PY{o+ow}{in} \PY{n}{soup}\PY{o}{.}\PY{n}{find\PYZus{}all}\PY{p}{(}\PY{l+s+s1}{\PYZsq{}}\PY{l+s+s1}{a}\PY{l+s+s1}{\PYZsq{}}\PY{p}{,} \PY{n}{href}\PY{o}{=}\PY{k+kc}{True}\PY{p}{)}\PY{p}{:}
            \PY{k}{if} \PY{n}{doc}\PY{o}{.}\PY{n}{get\PYZus{}text}\PY{p}{(}\PY{p}{)} \PY{o}{==} \PY{l+s+s1}{\PYZsq{}}\PY{l+s+s1}{View/Open}\PY{l+s+s1}{\PYZsq{}}\PY{p}{:}
                \PY{n}{links}\PY{o}{.}\PY{n}{append}\PY{p}{(}\PY{n}{doc}\PY{p}{[}\PY{l+s+s1}{\PYZsq{}}\PY{l+s+s1}{href}\PY{l+s+s1}{\PYZsq{}}\PY{p}{]}\PY{p}{)}

        \PY{c+c1}{\PYZsh{}Recuperando e gravando arquivo PDF}
        \PY{n}{url} \PY{o}{=} \PY{l+s+s1}{\PYZsq{}}\PY{l+s+s1}{http://repositorio.ufba.br}\PY{l+s+s1}{\PYZsq{}} \PY{o}{+} \PY{n}{links}\PY{p}{[}\PY{l+m+mi}{0}\PY{p}{]}
        \PY{n}{pdf} \PY{o}{=} \PY{n}{requests}\PY{o}{.}\PY{n}{get}\PY{p}{(}\PY{n}{url}\PY{p}{,} \PY{n}{proxies} \PY{o}{=} \PY{n}{proxies}\PY{p}{)}
        \PY{n}{filename} \PY{o}{=} \PY{n}{tese}\PY{p}{[}\PY{l+m+mi}{1}\PY{p}{]}\PY{p}{[}\PY{l+s+s1}{\PYZsq{}}\PY{l+s+s1}{PDF\PYZus{}ID}\PY{l+s+s1}{\PYZsq{}}\PY{p}{]} \PY{o}{+} \PY{l+s+s1}{\PYZsq{}}\PY{l+s+s1}{.pdf}\PY{l+s+s1}{\PYZsq{}}
        \PY{k}{with} \PY{n+nb}{open}\PY{p}{(}\PY{n}{filename}\PY{p}{,} \PY{l+s+s1}{\PYZsq{}}\PY{l+s+s1}{wb}\PY{l+s+s1}{\PYZsq{}}\PY{p}{)} \PY{k}{as} \PY{n}{f}\PY{p}{:}
            \PY{n}{f}\PY{o}{.}\PY{n}{write}\PY{p}{(}\PY{n}{pdf}\PY{o}{.}\PY{n}{content}\PY{p}{)}
    \PY{k}{except}\PY{p}{:}
        \PY{k}{pass}
\end{Verbatim}
\end{tcolorbox}

    \begin{Verbatim}[commandchars=\\\{\}]
UFBA\_\_15263
UFBA\_\_16019
UFBA\_\_16132
UFBA\_\_16133
UFBA\_\_16220
UFBA\_\_16253
UFBA\_\_16319
UFBA\_\_16473
UFBA\_\_16687
UFBA\_\_16914
UFBA\_\_16996
UFBA\_\_16997
UFBA\_\_17225
UFBA\_\_17228
UFBA\_\_17344
UFBA\_\_17986
UFBA\_\_18451
UFBA\_\_18607
UFBA\_\_18618
UFBA\_\_18667
UFBA\_\_18740
UFBA\_\_18744
UFBA\_\_18832
UFBA\_\_18834
UFBA\_\_18860
UFBA\_\_18862
UFBA\_\_19076
UFBA\_\_19079
UFBA\_\_19127
UFBA\_\_19128
UFBA\_\_19130
UFBA\_\_19147
UFBA\_\_19148
UFBA\_\_19156
UFBA\_\_19177
UFBA\_\_19368
UFBA\_\_19402
UFBA\_\_19526
UFBA\_\_19544
UFBA\_\_19591
UFBA\_\_19592
UFBA\_\_19622
UFBA\_\_19679
UFBA\_\_19682
UFBA\_\_19683
UFBA\_\_20243
UFBA\_\_20271
UFBA\_\_20274
UFBA\_\_20282
UFBA\_\_20378
UFBA\_\_20628
UFBA\_\_21240
UFBA\_\_21329
UFBA\_\_21334
UFBA\_\_21336
UFBA\_\_21463
UFBA\_\_21480
UFBA\_\_21489
UFBA\_\_21511
UFBA\_\_21526
UFBA\_\_21528
UFBA\_\_21560
UFBA\_\_21569
UFBA\_\_21646
UFBA\_\_21689
UFBA\_\_21973
UFBA\_\_21975
UFBA\_\_21977
UFBA\_\_22498
UFBA\_\_22530
UFBA\_\_22940
UFBA\_\_22953
UFBA\_\_23359
UFBA\_\_23365
UFBA\_\_23427
UFBA\_\_23450
UFBA\_\_23487
UFBA\_\_23493
UFBA\_\_23512
UFBA\_\_23879
UFBA\_\_23901
UFBA\_\_23944
UFBA\_\_24222
UFBA\_\_24311
UFBA\_\_24314
UFBA\_\_24316
UFBA\_\_24320
UFBA\_\_24375
UFBA\_\_24391
UFBA\_\_24393
UFBA\_\_24547
UFBA\_\_24583
UFBA\_\_24618
UFBA\_\_24700
UFBA\_\_24701
UFBA\_\_24702
UFBA\_\_24945
UFBA\_\_25285
UFBA\_\_25347
UFBA\_\_25595
UFBA\_\_25686
UFBA\_\_25687
UFBA\_\_25762
UFBA\_\_25773
UFBA\_\_25918
UFBA\_\_26114
UFBA\_\_26116
UFBA\_\_26117
UFBA\_\_26123
UFBA\_\_26124
UFBA\_\_26278
UFBA\_\_26279
UFBA\_\_26828
UFBA\_\_27030
UFBA\_\_27103
UFBA\_\_27130
UFBA\_\_27519
UFBA\_\_27520
UFBA\_\_27776
UFBA\_\_28214
UFBA\_\_28655
UFBA\_\_28656
UFBA\_\_28753
UFBA\_\_28907
UFBA\_\_28929
UFBA\_\_28937
UFBA\_\_28939
UFBA\_\_28940
UFBA\_\_28943
UFBA\_\_29380
UFBA\_\_29383
UFBA\_\_29385
UFBA\_\_10002
UFBA\_\_10007
UFBA\_\_10015
UFBA\_\_10023
UFBA\_\_10024
UFBA\_\_10058
UFBA\_\_10146
UFBA\_\_10164
UFBA\_\_10205
UFBA\_\_10529
UFBA\_\_10723
UFBA\_\_10809
UFBA\_\_10817
UFBA\_\_11136
UFBA\_\_11262
UFBA\_i\_1152
UFBA\_\_11657
UFBA\_\_12153
UFBA\_\_13139
UFBA\_\_13195
UFBA\_\_13200
UFBA\_i\_7169
UFBA\_i\_7170
UFBA\_i\_7172
UFBA\_i\_7176
UFBA\_i\_7180
UFBA\_i\_7181
UFBA\_i\_7446
UFBA\_i\_7447
UFBA\_i\_7611
UFBA\_i\_7612
UFBA\_i\_7640
UFBA\_i\_7683
UFBA\_i\_7689
UFBA\_i\_7959
UFBA\_i\_8049
UFBA\_i\_8053
UFBA\_i\_8088
UFBA\_i\_8163
UFBA\_i\_8874
UFBA\_i\_8885
UFBA\_i\_8890
UFBA\_i\_8905
UFBA\_i\_8910
UFBA\_i\_8924
UFBA\_i\_8930
UFBA\_i\_8933
UFBA\_i\_8936
UFBA\_i\_9510
UFBA\_i\_9511
UFBA\_i\_9714
UFBA\_i\_9867
UFBA\_i\_9953
UFBA\_i\_9962
UFBA\_i\_9982
\end{Verbatim}

    \begin{tcolorbox}[breakable, size=fbox, boxrule=1pt, pad at break*=1mm,colback=cellbackground, colframe=cellborder]
\prompt{In}{incolor}{ }{\hspace{4pt}}
\begin{Verbatim}[commandchars=\\\{\}]

\end{Verbatim}
\end{tcolorbox}


    % Add a bibliography block to the postdoc
    
    
    
    \end{document}
